% ---------------------------------- AES ---------------------------------------

\documentclass[a4paper,12pt]{book}

%TODO: aggiungere un Abstract? (L'objective (l'obiettivo) di questa ricerca è ...)
%TODO: aggiungere bibliografia.
%TODO: aggiungere indice analitico alla fine del "libro".

%TODO: Aggiungere capitolo: "Implementazione" (implementazione basilare) in C++ e/o Java.

% ----------------------------- BEGIN PREAMBLE ---------------------------------------

\usepackage{lmodern}
\usepackage{alltt, fancyvrb, url}
\usepackage{float}
\usepackage{graphicx}
\usepackage[utf8]{inputenc}
\usepackage{titlesec} %porre titlesec prima di hyperref per evitare il warning "The anchor of a bookmark and its parent's must not(hyperref) be the same".
\usepackage[breaklinks]{hyperref} %porre hyperref infondo al preamble perché altrimenti potrebbe dare warning tipo: "The anchor of a bookmark and its parent's must not(hyperref) be the same" dovuto al fatto che titlesec si trova dopo hyperref.
\usepackage{amsmath,amssymb,amsthm}

\usepackage{geometry} % [showframe] % serve per mostrare il margine, è da rimuovere quando ho finito di controllare il margine. %TODO: remove showframe %TODO: sistemare margini se uso geometry.

\usepackage[italian]{babel}

\usepackage[italian]{cleveref}

\usepackage{comment}
\usepackage[nopatch={eqnum,footnote}]{microtype} %[nopatch={eqnum,footnote}] aggiunto per evitare il warning "patching footnotes failed"
\usepackage{fancyhdr}

\usepackage[scaled=.92]{helvet}
\usepackage[T1]{fontenc}

\usepackage{lscape}

\usepackage{subcaption}

% ------ title -------------------
\usepackage[svgnames]{xcolor}
\ifpdf
\usepackage{pdfcolmk}
\fi
%% check if using xelatex rather than pdflatex
\ifxetex
\usepackage{fontspec}
\fi
%% drawing package
\usepackage{tikz}
%% for dingbats
\usepackage{pifont}
\providecommand{\HUGE}{\Huge}% if not using memoir
\newlength{\drop}
% ------ end title -------------------

% ------ chapter style -----------
\usepackage{kpfonts}
\usepackage{xcolor,calc, blindtext}
\definecolor{chaptercolor}{gray}{0.8}
% ------ end chapter style -----------

% ----- font family ------------------

%\usepackage{tgbonum}
%\fontfamily{cmss}\selectfont

\fontencoding{T1}
\fontfamily{garamond}
\fontseries{m}
%\fontshape{it}
%\fontsize{12}{15}
\selectfont %da warning forse perché viene usato \bfseries\scshape

%\usepackage{utopia} %utopia è obsoleto; usare fourier al posto.
% ----- end font family ------------------

% ------ section ---------

% ------ end section ---------

\usepackage{longtable}

% ---- ornamenti (distaccamenti) ---------

% per gli ornamenti, servono per distaccare parti del testo.
\usepackage{fourier-orns}

\newcommand{\fleuron}{
	\par\nopagebreak
	\parbox{\linewidth}{
		\centering\bigskip\aldineleft\par\bigskip
	}
}

\newcommand{\ornament}{
	\par\nopagebreak
	\parbox{\linewidth}{
		\centering\bigskip$\ast$\par$\ast\quad\ast$\par\bigskip
	}
}

% ---- ornamenti (distaccamenti) end ---------

% ----- firma (signature) begin ----------------

%\usepackage{emerald} %questo non lo trova
\usepackage{frcursive}
\usepackage{inslrmin}

\usepackage{soul} % per la sottolineatura.
\usepackage{setspace} % prima aveva anche l'option [doublespacing], ma dava errore.

% ----- firma (signature) end ----------------

% ----- per molteplici colonne begin ----------------

\usepackage{multicol}

% ----- per molteplici colonne end ----------------

% ----- empty page command begin -------------

\newcommand\blankpage{% comando pagina vuota
	\clearpage
	\begingroup
	\null
	\thispagestyle{empty}%
	\addtocounter{page}{-1}%
	\hypersetup{pageanchor=false}%
	\clearpage
	\endgroup
}

% ----- empty page command end -------------

% ----- code ----------------
%TODO: remove

\usepackage{listingsutf8}

\definecolor{codegreen}{rgb}{0,0.6,0}
\definecolor{codegray}{rgb}{0.5,0.5,0.5}
\definecolor{codepurple}{rgb}{0.58,0,0.82}
\definecolor{backcolour}{rgb}{0.95,0.95,0.92}
\definecolor{myblue}{rgb}{0.3, 0.6, 0.8}
\definecolor{myblue2}{rgb}{0.11, 0.19, 0.46}
\definecolor{myblue3}{rgb}{0.27, 0.55, 0.81}

% VS2017 C++ color scheme
\definecolor{clr-background}{RGB}{255,255,255}
\definecolor{clr-text}{RGB}{0,0,0}
\definecolor{clr-string}{RGB}{163,21,21}
\definecolor{clr-namespace}{RGB}{0,0,0}
\definecolor{clr-preprocessor}{RGB}{128,128,128}
\definecolor{clr-keyword}{RGB}{0,0,255}
\definecolor{clr-type}{RGB}{43,145,175}
\definecolor{clr-variable}{RGB}{0,0,0}
\definecolor{clr-constant}{RGB}{111,0,138} % macro color
\definecolor{clr-comment}{RGB}{0,128,0}

\definecolor{weborange}{RGB}{255,165,0}

\lstdefinestyle{VS2017}{
	backgroundcolor=\color{clr-background}, % oppure darkgray o darkgrey %TODO: mettere quello classico.
	basicstyle=\color{clr-text}, % any text
	stringstyle=\color{clr-string},
	identifierstyle=\color{clr-variable}, % just about anything that isn't a directive, comment, string or known type
	commentstyle=\color{clr-comment},
	directivestyle=\color{clr-preprocessor}, % preprocessor commands
	% listings doesn't differentiate between types and keywords (e.g. int vs return)
	% use the user types color
	keywordstyle=\color{clr-type},
	keywordstyle={[2]\color{clr-constant}}, % you'll need to define these or use a custom language
	breaklines=true,
	showspaces=false,
	showstringspaces=false,
	%otherkeywords={>,<,.,;,-,!,=,~},
	morekeywords={\#, std, std::cout, cout, std::endl, endl, ::, ifndef, define, endif, pragma, override, decltype, noexcept, alignas, alignof, constexpr, assert, nullptr}, % \# non funziona.
	%keywordstyle=\color{weborange},
	tabsize=4
}

\lstdefinestyle{mystyle}{
	backgroundcolor=\color{backcolour},   
	commentstyle=\color{codegreen},
	keywordstyle=\color{myblue2}, % prima era magenta
	numberstyle=\tiny\color{codegray},
	stringstyle=\color{codepurple},
	basicstyle=\ttfamily\footnotesize,
	breakatwhitespace=false,         
	breaklines=true,                 
	captionpos=b,                    
	keepspaces=true,                 
	numbers=left,                    
	numbersep=5pt,                  
	showspaces=false,                
	showstringspaces=false,
	showtabs=false,                  
	tabsize=2
	%classoffset=1, % starting new class
	%otherkeywords={>,<,.,;,-,!,=,~},
	%morekeywords={>,<,.,;,-,!,=,~},
	%keywordstyle=\color{red},
	%classoffset=0,
}

%\lstset{inputencoding=utf8/latin1} % Questo non andava.

\lstset{language=C++,texcl=true} % questo mi ha fixato il problema delle lettere
% accentate nei commenti, ma non nelle stringhe " " del codice.

% Altrimenti sempre per il problema delle lettere accentat, si può usare questo:
% Anzi questo mi aiuta per le stringhe nel codice " ".
%\begin{comment}
\lstset{
	literate=%
	{á}{{\'a}}1
	{à}{{\`a}}1
	{ã}{{\~a}}1
	{é}{{\'e}}1
	{è}{{\`e}}1
	{ê}{{\^e}}1
	{í}{{\'i}}1
	{ó}{{\'o}}1
	{õ}{{\~o}}1
	{ú}{{\'u}}1
	{ü}{{\"u}}1
	{ù}{{\`u}}1
	{ç}{{\c{c}}}1
	{¡}{{\'!}}1
}
%\end{comment}

\lstset{style=VS2017}

%frame (ovvero un box) attorno al codice
\begin{comment}
	\usepackage[most]{tcolorbox}
	\usepackage{inconsolata}
	
	\newtcblisting[auto counter]{listingframe}[2][]{sharp corners, 
		fonttitle=\bfseries, colframe=gray, listing only, 
		listing options={basicstyle=\ttfamily,language=java}, 
		title=Listing \thetcbcounter: #2, #1}
\end{comment}

% ------ end code -----------

% hyperref settings
\hypersetup{
	colorlinks=true,
	linkcolor=black, %blue
	filecolor=magenta,      
	urlcolor=cyan,
	pdftitle={Sharelatex Example},
	bookmarks=true, %commentato per via del warning
	pdfpagemode=FullScreen,
}

% ----------------------------- END PREAMBLE -----------------------------------------

% ----------------------------- BEGIN CHAPTER STYLE ----------------------------------

% ================== Chapter Style 1 ================================================

%\definecolor{gray75}{gray}{0.75}
%\newcommand{\hsp}{\hspace{20pt}}

%\titleformat{\chapter}[hang]{\Huge\bfseries}{\thechapter\hsp\textcolor{gray75}{|}\hsp}{0pt}{\Huge\bfseries}

% ================== Chapter Style 2 ================================================

%Options: Sonny, Lenny, Glenn, Conny, Rejne, Bjarne, Bjornstrup
\usepackage[Lenny]{fncychap}

% ----------------------------- END CHAPTER STYLE ------------------------------------


% ----------------------------- BEGIN SECTION STYLE ----------------------------------

\titleformat
{\section} % command
[display] % shape
{\bfseries\Large\itshape} % format
{} % label
{0.5ex} % sep
{
	\vspace{1ex} % prima era 1ex
	\rule{\textwidth}{2pt} % 2pt
	\centering
} % before-code
[
\vspace{-2ex}% 2ex
\rule{\textwidth}{1.5pt} % prima era 1.5 col font classico.
] % after-code

% ----------------------------- END SECTION STYLE ------------------------------------

% ----------------------------- BEGIN SUBSECTION STYLE -------------------------------

\titleformat{\subsection}{\centering\bfseries\Large\itshape}{}{}{}

% ----------------------------- END SUBSECTION STYLE ---------------------------------

% ----------------------------- BEGIN SUBSUBSECTION STYLE ----------------------------

\titleformat{\subsubsection}{\centering\bfseries\large\itshape}{}{}{}

% ----------------------------- END SUBSUBSECTION STYLE ------------------------------

% ----------------------------- BEGIN PARAGRAPH STYLE --------------------------------

\titleformat{\paragraph}{\centering\bfseries\normalsize}{}{}{}

% ----------------------------- END PARAGRAPH STYLE ----------------------------------

% ----------------------------- BEGIN TABLE OF CONTENTS ------------------------------

\setcounter{tocdepth}{4}
\setcounter{secnumdepth}{1}

% ----------------------------- END TABLE OF CONTENTS --------------------------------

%\usepackage{enumitem}
%\setlist[description]{style=nextline}
%\leavevmode

\newcommand*{\titleRF}{\begingroup% Robert Frost, T&H p 149
	%\FSfont{5bp} % FontSite Bergamo (Bembo)
	%49
	\drop = 0.2\textheight
	\centering %TODO: questo centering da problemi; uncomment
	\vfill
	{\Huge Advanced Encryption Standard}\\[\baselineskip]
	{\Huge AES}\\[\baselineskip]
	{\large Luca Rengo}\\[0.5\drop]
	{\Large }%\\[0.5\baselineskip] %TODO: \plogo da problemi; questo \\ dava l'errore: "there is no line to end here"
	{\Large The Publisher}\par %TODO: al posto di The Publisher scrivere qualcos'altro
	{\large \scshape year 2022} 
	\vfill\null
	\endgroup}

% ----------------------------- END TITLE PAGE ---------------------------------------

% ----------------------------- BEGIN DOCUMENT ---------------------------------------

\begin{document}
	
	\frontmatter
	
	\titleRF
	
	\tableofcontents
	
	% \input: import the commands from filename.tex to target file.
	
	% \include: does a \clearpage and does an \input.
	
	% \import: needs \usepackage{import} and it's used only when the imported files needs the path. \import{path}{filename}
	
	\mainmatter
	
	%TODO: magari non serve l'introduzione per ogni capitolo? %TODO: remove
	
	%TODO: Abstracts
	
	% ================================ STORIA DI AES =======================================

\chapter{Storia di AES}

% ======================================================================================


% ---------------------------- SECTION: INTRODUZIONE -----------------------------------

%\newpage

\section{Introduzione}

%TODO: Che cos'è AES? Com'è nato? Chi l'ha inventato e perché?

%TODO: AES è ..., è stato inventato da ..., ecc. 

% ---------------------------- SECTION: STORIA DI AES ----------------------------------

\newpage

\section{Storia di AES} %TODO: rename "Breve storia di AES"

%TODO: Partire dal DES, dai problemi del DES e del triplo DES
%TODO: Nist 1997
%TODO: I concorrenti di AES: serpent, twofish, ecc. e dire per quale motivo è stato scelto Rijndael come AES alla fine.

% -------------------------------- FINE CAPITOLO ---------------------------------------
	
	% ================================ L'ALGORITMO =========================================

\chapter{L'Algoritmo} %TODO: Rinominare in "Una panoramica sull'Algoritmo" o in realtà quello è per una sezione?

%TODO: Le varie sezioni: sub bytes, Shift Rows, mix columns, Add Round Key
%TODO: aggiungere immagini

%TODO: perché l'ultimo round è differente?

%TODO: 1. KeyExpansion (Rijndael Key Scheduler/AES Key Scheduler)
%TODO: 1. AddRoundKey (con lo XOR)

%TODO: 1. SubBytes (S-Box)
%TODO: 2. ShiftRows (1a riga niente, 2a riga = shift a sinistra di 1, 3a riga = shift di 2, 4a riga = shift di 3) (lo shift è come una ROL in x86)
%TODO: 3. MixColumns (Rijndael Key Scheduler/AES Key Scheduler)
%TODO: 4. AddRoundKey (con lo XOR)

%TODO: ShiftRows e MixColumns per la "Confusione e Diffusione" (teoria di Shannon).

%TODO: 10 rounds (fasi/cicli) per chiavi di 128-bit
%TODO: 12 rounds (fasi/cicli) per chiavi di 192-bit
%TODO: 14 rounds (fasi/cicli) per chiavi di 256-bit

% ======================================================================================

% ---------------------------- SECTION: INTRODUZIONE -----------------------------------

%\newpage

\section{Introduzione}

%In questo capitolo, tratteremo il funzionamento dell'algoritmo di AES con una panoramica dall'alto, per poi affrontare nel prossimo capitolo, più in dettaglio, la matematica dietro a questo. %TODO: "dietro a questo" suona male, meglio trovare qualcos'altro.

%TODO: In questo capitolo affronteremo il funzionamento dell'algoritmo di AES più nel dettaglio/con una visione dall'alto, per poi trattare nel prossimo capitolo la matematica dietro all'\emph{Advanced Encryption Standard}.

% ---------------------------- SECTION:  ----------------------------------

\newpage

\section{} %TODO: Le tre idee/concetti/principi dietro la Crittografia

%TODO: Alla base della crittografia, ci sono due importanti proprietà/principi dei cifrari a chiave simmetrica, elaborati dal padre della teoria dell'informazione, Claude Elwood Shannon, ovvero: \emph{diffusione} e \emph{confusione}.

%TODO: Il principio/proprietà della \emph{confusione} nasconde/vela la connessione tra il messaggio originale e il messaggio/testo cifrato. [per esempio cifrario di cesare (con annessa immagine volendo)]

%TODO: Il principio/proprietà di \emph{diffusione}, invece, riguarda il riordino/ridistribuzione/scombussolamento della posizione delle lettere/caratteri nel/del messaggio/testo. [un semplice esempio potrebbe essere la trasposizione delle colonne in una matrice oppure non lo scrivo.]

%TODO: Un altro concetto/astrazione? molto importante è quello della \emph{segretezza nella chiave}, ovvero che l'algoritmo alla base del cifrario è conosciuto, è pubblico, ma la sola conoscenza di questo non basta/non è sufficiente per poter ottenere/attingere/conseguire l'accesso alle informazioni, perché per poterlo raggiungere/attingere/acquisire/conseguire avraì bisogno/sarà necessario conoscere la chiave segreta. [volendo aggiungere altro sulla teoria dell'informazione, su Shannon, sull'importanza, ecc. Magari aggiungere anche un'immagine.]

%TODO: [aggiungere una o più immagini sulla teoria di Shannon]

% ---------------------------- SECTION:  ----------------------------------

\section{} %TODO: Una panoramica dell'Algoritmo/Una panoramica sull'Algoritmo (meglio una panoramica sull'algoritmo come nome!) oppure "L'Algoritmo in breve" o "Semplice Overview dell'Algoritmo".

%TODO: I dati di input vengono caricati in una matrice 4x4, anche chiamata \emph{state matrix} (matrice di stato), ogni cella contiene/rappresenta 1 byte di informazione, su questa compiamo diverse operazioni: sub-bytes (sostituzione dei bytes), shift rows (spostamento delle righe), mix columns (mescolamento delle colonne), add round key (aggiunta della chiave del round) per un numero di volte/rounds pari/in base alla grandezza della chiave. [immagine key size / Rounds]

%TODO:  Nel primo round svolgiamo uno XOR (bit-a-bit?) tra il messaggio/testo d'input e la chiave segreta. [inserire immagine tabella verità XOR]

%TODO: Lo XOR (EXclusive-OR) bit-a-bit è un'operazione di mascheratura dei bit, dove se i due bit di input sono diversi, allora genererà/produrrà un 1 in uscita, altrimenti se sono uguali, uno zero.

%TODO: Sezione/Subsection: Perché lo XOR è usato in crittografia?
%TODO: itemize?
%TODO: Lo XOR non fa fuoriuscire/fuori esce/leak informazioni (sull'input inziiale/originale?)
%TODO: Lo XOR è \emph{involutory function}, se lo applichi due volte, riottieni il testo originale.
%TODO: L'output dello XOR dipende da entrambi gli input. Non è così per le altre operazioni. (Scrivere: AND, OR, NOT, ecc.?)

%TODO: piccolo siparietto per tornare sull'argomento della sezione. \fleuron o \ornament

%TODO: Fare una subsection?: titolo: "Come vengono ottenute le chiavi per ogni round?" oppure no?
%TODO: Per poter eseguire/elaborare i rounds, l'algoritmo ha bisogno di molte chiavi [una per round?], queste vengono tutte derivate dalla chiave iniziale. (volendo scrivere: Questo è uno dei motivi per cui AES viene criticato, perché tutte le chiavi vengono derivate dalla chiave originale).

%TODO: Il procedimento è questo: [mettere delle immagini a riguardo]

%TODO: enumerate: 
%TODO: 1) Sposta la prima cella dell'ultima colonna della precedente chiave, in fondo/basso alla colonna.
%TODO: 2) Ogni byte viene eseguito in una substituion box [che lo mapperà in qualcos' altro].
%TODO: 3) Poi viene effettuato uno XOR tra la colonna e un \emph{round costant} (una costante di round) che è diversa per ogni round.
%TODO: 4) Infine viene realizzato uno XOR con la prima colonna della precedente chiave.

%TODO: Per quanto riguarda le altre colonne, vengono semplicemente eseguiti degli XOR con la stessa colonna della precedente chiave. [magari spiegare meglio] [magari mettere un'immagine a riguardo] [eccetto per le chiavi a 256 bit che sono un po' più complicate] [approfondire il perché siano più complicate]

%TODO: Dopo aver ottenuto queste chiavi vengono compiuti i vari rounds.

%TODO: Per ogni round, eseguiamo questi passaggi tranne per l'ultimo dove nonn effettuiamo il passaggio delle "Mix columns", perché non aumenterebbe la sicurezza e semplicemente rallenterebbe: [magari approfondire il perché]

%TODO: itemize: 

%TODO: \textbf{Sub-bytes} oppure \underline{\emph{Sub-bytes}}: ...

%TODO: \textbf{Shift Rows} oppure \underline{\emph{Shift Rows}}:

%TODO: \textbf{Mix Columns} oppure \underline{\emph{Mix Columns}}:

%TODO: \textbf{Add Round Key} oppure \underline{\emph{Add Round Key}}:

% -------------------------------- FINE CAPITOLO --------------------------------------- %TODO: oppure chiamarlo: "Meccanismo"/"Il Meccanismo"?
	
	% ================================ LA MATEMATICA DIETRO AES ===========================

\chapter{La Matematica dietro AES}

%TODO: matrici 4x4
%TODO: Campo finito (campo di Galois). (p^n elementi)
%TODO: trasformazione affine invertibile.
%TODO: punto fisso (in matematica), punti fissi opposti.

%TODO: S-Box
%TODO: Rijndael Key Scheduler/AES Key Scheduler

%TODO: numero colonne = numero bit blocco in input / 32

% ======================================================================================

%TODO: 
%TODO: Parto spiegando che cos'è il campo finito di Galois GF(2^8)

%TODO: Cos'è un gruppo abelliano: gruppo con operazione con proprietà commutativa 
%TODO: Che cos'è un monoide: struttura algebrica con proprietà associativa e elemento neutro
%TODO (Che cos'è un gruppo)

%TODO: cos'è un anello: 3 assiomi degli anelli: gruppo abelliano rispetto all'addizione, monoide rispetto alla moltiplicazione, moltiplicazione distribuita rispetto all'addizione.
%TODO: spiegare omomorfismo e isomorfismo negli anelli?
%TODO: (Che cos'è un assioma)
%TODO: differenza tra campo e anello ed esempi a riguardo. (uno sull'anello e uno sui campi)
%TODO: anello finito: è un anello con un numero finito di elementi

%TODO: Teoria di Galois, Teorema di Abel-Ruffini?, Teorema fondamentale della Teoria di Galois?

%TODO: spiego le varie operazioni possibili in questo campo e come si eseguono:
%TODO: i numeri si possono rappresentare come polinomi, ovvero, per esempio: il numero esadecimale 0x53, ovvero 83 in decimale o 01010011 in binario può essere rappresentato come x^6 + x^4 + x + 1 perché 01010011 (1 * 2^6, 1 * 2^4, 1 * 2^1, 1 * 2^0 <=> x^6 + x^4 + x + 1)
%TODO: Somma e Sottrazione si fanno con lo XOR. 
%TODO: mostrare un esempio
%TODO: Moltiplicazione: modulo x^8 + x^4 + x^3 + x + 1 e il perché
%TODO: mostrare un esempio di moltiplicazione
%TODO: esponenziazione
%TODO: esempio o tabella degli esponenti
%TODO: Logaritmi
%TODO: esempio o tabella dei logaritmi
%TODO: Divisione
%TODO: esempio della divisione
%TODO: trovare il polinomio inverso

%TODO: Magari fino alla Teoria fondamentale potrei spiegare, poi passiamo ad AES.
%TODO: Teorema fondamentale della teoria di Galois?
%TODO: Estensione di Galois?
%TODO: Estensione Algebrica?
%TODO: Estensione di campi?
%TODO: Gruppo di Galois?

%TODO: la moltiplicazione viene usata nella fase "mix columns".
%TODO: l'inverse multiplication viene usata nella s-box.


%TODO: spiego le varie operazioni dell'algoritmo come vengono eseguite in termini matematici.
%TODO: l's-box g e f => f(g(x))
%TODO: round constants nella key expansion
%TODO: Mix Columns: ogni colonna come polinomio, la moltiplico per un altro polinomio e recupero il resto dopo averla divisa per x^4 + 1, tutto questo è una moltiplicazione tra matrici.
%TODO: e altro.

%TODO: volendo aggiungere dei riferimenti, ovvero dei \label e \ref alla parti che spiego perché sono connesse.

%TODO: Aggiungere delle immagini volendo.

% ---------------------------- SECTION: INTRODUZIONE ----------------------------------

%\newpage

\section{Introduzione}

%TODO: In questo capitolo tratteremo più a fondo la matematica che sta dietro questo algoritmo. Per poterlo fare avremo bisogno di introdurre alcune strutture matematiche come: i gruppi, gli anelli e i campi, dopodiché ci soffermeremo su uno/un specifico/determinato campo, quello di Galois, ne mostreremo le operazioni possibili e come vengono usufruite/utilizzate nei steps/passi mostrati/trattati nel capitolo degli algoritmi.
%TODO: da sistemare l'introduzione. 
\textsf{\small In questo capitolo tratteremo più a fondo la matematica che sta dietro questo algoritmo. Per poterlo fare avremo bisogno di introdurre alcune strutture matematiche come: i gruppi, gli anelli e i campi, dopodiché ci soffermeremo su uno specifico campo, quello di Galois, ne mostreremo le operazioni possibili e come ne usufruiremo nei passi trattati nel capitolo degli algoritmi.}

% ------------------------- SECTION: GRUPPI, ANELLI E CAMPI  --------------------------

\newpage

%TODO: volendo spiegare cos'è un assioma.

\section{Gruppi, Anelli e Campi} %TODO: oppure "Gruppi, Anelli e Campi: Le differenze" oppure "Campi, Gruppi e Anelli" o "Campi, Anelli e Gruppi".

\textsf{\small Prima di introdurre il campo di Galois, facciamo chiarezza tra i Gruppi, gli Anelli e i Campi.}

%TODO: Prima di introdurre il campo di Galois, facciamo chiarezza/differenziamo i/tra Gruppi, gli Anelli e i Campi.

%TODO: o spiegare i gruppi, gli anelli e i campi prima del campo di Galois o dopo! Li metto in una sezione a parte o come subsections di "Il campo di Galois"?

\subsection{Gruppo}

%TODO: Un gruppo è una struttura algebrica composta da un'insieme vuoto \emptyset e un'operazione binaria (+ e \cdot) per l'elemento neutro e per l'elemento inverso.
%TODO: Il gruppo deve soddisfare tre assiomi:
%TODO: itemize o enumerate
%TODO: I. proprietà associativa: dati a, b, c;  (a * b) * c = a * (b * c) %TODO: oppure usare x, y, z? Usare \cdot o *
%TODO: II. esistenza dell'elemento neutro: elemento neutro \emph{e} tale che a \cdot e = e \cdot a = a %TODO: a o alfa?
%TODO: III. esistenza dell'inverso: esiste a', detto inverso, tale che a \cdot a' = a' \cdot a = e
\textsf{\small Un gruppo è una struttura algebrica composta da un'insieme vuoto $\emptyset$ e un'operazione binaria ($+ \text{ e } \cdot$) per l'elemento neutro e per l'elemento inverso.}

\textsf{\small Il gruppo deve soddisfare tre assiomi:}

\begin{enumerate}[I.] %con package enumitem: [label=\Roman.]
	\item \textsf{\small proprietà associativa: dati a, b, c;  $(a \cdot b) \cdot c = a \cdot (b \cdot c)$}
	\item \textsf{\small esistenza dell'elemento neutro: elemento neutro \emph{e} tale che $a \cdot e = e \cdot a = a$}
	\item \textsf{\small esistenza dell'inverso: esiste a', detto inverso, tale che a $\cdot a' = a' \cdot a = e$}
\end{enumerate}

\subsubsection{Monoide}

%TODO: Il monoide è una struttura algebrica fornita/dotata dell'operazione binaria associativa e di un elemento neutro.
\textsf{\small Il monoide è una struttura algebrica fornita dell'operazione binaria associativa e di un elemento neutro.}

\subsubsection{Gruppo abeliano}

%TODO: È un gruppo che gode della proprietà commutativa.
\textsf{\small  È un gruppo che gode della proprietà commutativa.}

\subsection{Anello}

%TODO: Un anello è una struttura algebrica, dove sono definite due operazioni: + e \cdot (somma e prodotto) che rispetta le proprietà: commutativa, distributiva (della moltiplicazione rispetto all'addizione) e dove esiste un elemento neutro nell'addizione (a + 0 = 0 + a = a) e nella moltiplicazione (a \cdot 1 = 1 \cdot a = a).

%TODO: Quindi un anello è:
%TODO: itemize
%TODO: - 2 operazioni binarie + e \cdot che rispettano i 3 assiomi degli anelli:
%TODO: 	- gruppo abelliano rispetto all'addizione (proprietà commutativa).
%TODO: 	- monoide rispetto alla moltiplicazione (proprietà associativa ed elemento neutro)
%TODO: 	- moltiplicazione distributiva rispetto all'addizione.
\textsf{\small Un anello è una struttura algebrica, dove sono definite due operazioni: $+ \text{ e } \cdot$ (somma e prodotto) che rispetta le proprietà: commutativa, distributiva (della moltiplicazione rispetto all'addizione) e dove esiste un elemento neutro nell'addizione ($a + 0 = 0 + a = a$) e nella moltiplicazione ($a \cdot 1 = 1 \cdot a = a$).}

\textsf{\small Quindi, un anello è: }

\begin{itemize}
	\item \textsf{\small  due operazioni binarie $+ \text{ e } \cdot$ che rispettano i 3 assiomi degli anelli:}
	\begin{itemize}
		\item \textsf{\small gruppo abeliano rispetto all'addizione (proprietà commutativa).}
		\item \textsf{\small monoide rispetto alla moltiplicazione (proprietà associativa ed elemento neutro)}
		\item \textsf{\small moltiplicazione distributiva rispetto all'addizione.}
	\end{itemize}
\end{itemize}

\subsubsection{Anello commutativo}

%TODO: È un tipo di anello in cui l'operazione moltiplicativa gode della proprietà commutativa.
\textsf{\small È un tipo di anello in cui l'operazione moltiplicativa gode della proprietà commutativa.}

\subsubsection{Anello finito} %TODO: o altro oppure non serve come subsubsection

%TODO: Un anello finito è un anello che possiede un numero finito di elementi.
\textsf{\small Un anello finito è un anello che possiede un numero finito di elementi.}

%TODO: Aggiungere un immagine per riempire il vuoto? O aggiungere altro.

\subsection{Campo} %TODO: o "Campi"?

%TODO: Un campo \mathbb{K} è una struttura algebrica formata da un insieme non vuoto e due operazioni binarie + e \cdot in cui:
%TODO: l'insieme e l'operazione + sono un gruppo abelliano con elemento neutro
%TODO: \mathbb{K} \ {0} e l'operazione \cdot sono un gruppo abelliano con elemento neutro 1.
%TODO: la moltiplicazione è distributiva rispetto all'addizione.

%TODO: Un campo può essere definito anche come un anello commutativo dove ogni elemento non nullo possiede un inverso.
%TODO: O anche come un corpo commutativo alla moltiplicazione. %TODO: volendo fare una subsection per il corpo?
\textsf{\small Un campo $\mathbb{K}$ è una struttura algebrica formata da un insieme non vuoto e due operazioni binarie $+ \text{ e } \cdot$ in cui:}

\begin{itemize}
	\item \textsf{\small l'insieme e l'operazione + sono un gruppo abeliano con elemento neutro.}
	\item \textsf{\small $\mathbb{K} \hspace{.1cm} \backslash \hspace{.1cm} \{0\}$ e l'operazione $\cdot$ sono un gruppo abeliano con elemento neutro 1.}
	\item \textsf{\small la moltiplicazione è distributiva rispetto all'addizione.}
\end{itemize}

\textsf{\small Un campo può essere definito anche come un anello commutativo dove ogni elemento non nullo possiede un inverso.}
\textsf{\small O anche come un corpo commutativo alla moltiplicazione.}

\subsubsection{Campi vs Anelli} %TODO: o "Campo vs Anello"? %TODO: forse questa parte non serve? rimuovere?

%TODO: La differenza fra queste due strutture è questa, ovvero:
%TODO: Un anello è un gruppo abeliano rispetto alla prima operazione (+ addizione), ma non rispetto alla seconda, mentre, il campo è un gruppo abelliano rispetto a entrambe le operazioni.
\textsf{\small La differenza fra queste due strutture è questa, ovvero:}

\textsf{\small Un anello è un gruppo abeliano rispetto alla prima operazione (+ addizione), ma non rispetto alla seconda, mentre, il campo è un gruppo abeliano rispetto a entrambe le operazioni.}

\textsf{\small Di seguito, alcuni esempi:}

\textsf{\small Esempio: L'anello dei numeri interi ($\mathbb{Z}, +, \cdot$) è un gruppo abeliano rispetto all'addizione: 5 + (-5) = 0. Ma non è un gruppo abeliano rispetto alla moltiplicazione, perché non esiste un inverso di un numero intero n tale che $n \cdot n^{-1} = 1, 5 \cdot \frac{1}{5} \underset{}{\notin \mathbb{Z}} = 1$}

%TODO: Di seguito, alcuni esempi:
%TOOD: Esempio: L'anello dei numeri interi (\mathbb{Z}, +, \cdot) è un gruppo abelliano rispetto all'addizione: 5 + (-5) = 0. Ma non è un gruppo abelliano rispetto alla moltiplicazione, perché non esiste un inverso di un numero intero n tale che n \cdot n^{-1} = 1, 5 * $\underset{\frac{1}{5}}{\notin \mathbb{Z}} = 1$

\textsf{\small Esempio: Anello numeri frazionari è un campo: $(\mathbb{Q}, +, \cdot)$ è un gruppo abeliano rispetto all'addizione e alla moltiplicazione: $5 + (-5) = 0. \hspace{.3cm} 5 \cdot \frac{1}{5} = 1$}

%TODO: Esempio: Anello numeri frazionari è un campo: $(\mathbb{Q}, +, \cdot)$ è un gruppo abelliano rispetto all'addizione e alla moltiplicazione: 5 + (-5) = 0. 5 \cdot $\frac{1}{5} = 1$

\subsection{Estensione di campi}

\textsf{\small Essa è uno studio di coppie di campi, l'uno contenuto nell'altro. Una tale coppia prende il nome di \textbf{estensione di campi}. }

\textsf{\small Se L è un campo e K è un campo contenuto in L tale che le operazioni di campo in K sono le stesse di quelle in L, diciamo che K è un sottocampo di L, che L è un'estensione di K e che $L/K$ è un'\emph{estensione di campi}.} %TODO: \cite{} Wikipedia

%TODO: Altro?

\subsection{Estensione algebrica} %TODO: prima questo era prima dell'estensione di campi.

\textsf{\small L'estensione algebrica, in algebra astratta, una estensione di campi $L/K$ è detta algebrica se ogni elemento L è ottenibile come radice di un qualche polinomio a coefficienti in K.}

\textsf{\small Sia K un campo, una estensione è il dato di un altro campo L e di un omomorfismo iniettivo di K in L. Tramite l'omomorfismo K può essere visto come un sottocampo di L. L'estensione è generalmente indicata con la notazione $L/K$.} %TODO: \cite{} Wikipedia

\subsection{Gruppo di Galois}

\textsf{\small Il gruppo di Galois è un gruppo associato a un'estensione di campi, in particolare i gruppi associati a estensioni che sono di Galois.} %TODO: \cite{} Wikipedia

\subsection{Estensione di Galois}

\textsf{\small È un'estensione algebrica $E/F$ che soddisfa le condizioni descritte qui sotto. Il senso è che un'estensione di Galois ha un gruppo di Galois e obbedisce al teorema fondamentale della teoria di Galois.} %TODO: \cite{} Wikipedia

% ---------------------------- SECTION: IL CAMPO DI GALOIS  ---------------------------

\section{Il campo di Galois} %TODO: o "Il camppo finito"

\textsf{\small Il campo finito o anche denominato il campo di Galois, dal matematico francese Évariste Galois, è un campo che contiene un numero finito di elementi che ci permette di attuare operazioni su numeri a 8 bits.} %TODO: da riscrivere meglio

\textsf{\small Fornisce una connessione tra teoria di campi e teoria dei gruppi, attraverso il teorema fondamentale della teoria di galois.} %TODO: questo è da approfondire!

\textsf{\small Lo introdusse per studiare le radici dei polinomi e risolvere queste equazioni polinomiali in termini di gruppi di permutazione delle loro radici.}

\textsf{\small Un'equazione è risolvibile dai suoi radicali se le sue radici si possono esprimere da una formula che riguarda numeri interi, radici ennesime e le 4 operazioni aritmetiche basilari $(+, -, \cdot, /)$.}

\textsf{\small } %TODO: Questo generalizza il teorema di Abel-Ruffini che afferma che non c'è soluzioni in radicali alle equazioni polinomiali di grado 5 o superiori con coefficienti arbitrari. %TODO: questo è da mettere dopo l'introduzione al campo di Galois! E da semplificare.

\textsf{\small Ora arriviamo al motivo per cui ci interessa:} %TODO: Riscrivere meglio?

\textsf{\small Il più comune campo di Galois è il campo GF(256) o anche designato GF($2^8$) (GF = Galois Field) che ci permette di definire operazioni sono a livello di byte, dove i bits vengono interpretati come coefficienti dei polinomi del campo finito.}
%TODO: 

%TODO: Subsection? Le operazioni possibili nel campo di Galois
\subsection{Le operazioni del campo finito} %TODO: oppure "Le operazioni del campo di Galois" oppure "Le operazioni di GF(2^8)".

\textsf{\small Nel campo finito sono disponibili le quattro operazioni basilari dell'aritmetica, $(+, -, \cdot, /)$ e di conseguenza l'esponenziazione e i logaritmi.} %TODO: riscrivere la parte "e di conseguenza"?

\textsf{\small I numeri vengono rappresentati come polinomi di 8 bits, ovvero, per esempio il numero esadecimale 0x53, ovvero 83 in decimale o 01010011 in binario può essere rappresentato come $x^6 + x^4 + x + 1$ perché 01010011 $(1 \cdot 2^6, 1 \cdot 2^4, 1 \cdot 2^1, 1 \cdot 2^0 <=> x^6 + x^4 + x + 1)$. }

%TODO: oppure scegliere un altro numero! <----

\subsubsection{Addizione e sottrazione} %TODO: oppure non serve fare una subsubsection?

%TODO: potrei utilizzare i gli $\underbrace{}$
\textsf{\small Somma e sottrazione vengono eseguite tramite uno XOR.}

\textsf{\small Per esempio: prendiamo due polinomi: $x + 1 \text{ e } x^2 + 1$.}
\textsf{\small $x + 1$ equivale a $2^1 + 2^0$, ovvero in binario il numero 11.}
\textsf{\small $x^2 + 1$ equivale a $1 \cdot 2^2 + 0 \cdot 2^1 + 1 \cdot 2^0$, ovvero in binario al numero 101.}
\textsf{\small dopodiché eseguiamo, semplicemente, uno XOR tra i due polinomi.}
\textsf{\small  $011 \oplus 101$ oppure $x + 1 \oplus x^2 + 1 = 110 \text{ o } x^2 + x$}

\subsubsection{Moltiplicazione}

\textsf{\small Per la moltiplicazione è un pochino più complicato.} %TODO: filino/è un pochino

\textsf{\small Avvenuta l'operazione, è necessario compiere un'operazione di modulo con il polinomio $x^8 + x^4 + x^3 + x + 1$. } %TODO: effettuare/compiere (magari spiegare il perché)

\textsf{\small per esempio $0x53 \cdot 0xCA = 0x01$, perché: } %TODO: cambiare numeri 0x53 e 0xCA

\textsf{\small $0x53 = 83 = 1010011 = x^6 + x^4 + x + 1$}

\textsf{\small $ 0xCA = 202 = 11001010 = x^7 + x^6 + x^3 + x $} \\

\textsf{\small $ (x^6 + x^4 + x + 1)(x^7 + x^6 + x^3 + x) $}

\textsf{\small $ = (x^{13} + x^{12} + x^9 + x^7)  + (x^{11} + x^{10} + x^7 + x^5) + (x^8 + x^7 + x^4 + x^2) + (x^7 + x^6 + x^3 + x)$}

\textsf{\small $ = x^{13} + x^{12} + x^{11} + x^{10} + x^9 + x^8 + x^6 + x^5 + x^4 + x^3 + x^2 + x$}

\textsf{\small $ x^{13} + x^{12} + x^{11} + x^{10} + x^9 + x^8 + x^6 + x^5 + x^4 + x^3 + x^2 + x \text{ modulo } x^8 + x^4 + x^3 + x + 1 = 1$.}

\subsubsection{Esponenziazione} %TODO: "Potenza"?

\textsf{\small L'esponenziazione è semplicemente la ripetizione della moltiplicazione.}

\textsf{\small Alcuni numeri possiedono la proprietà di attraversare tutti i 255 numeri non-zero del campo quando vengono moltiplicati per se stessi. Questi numeri vengono chiamati \emph{generatori}.} %TODO: attraversare/essere in grado di attraversare; non-zero/non-zeri; denominati/chiamati

\textsf{\small Possiamo raccogliere questi numeri in una tabella, la tabella dei generatori.}

%TODO: volendo mostrare la tabella in un'immagine. MOSTRARE LA TABELLA.

\subsubsection{Logaritmi}

\textsf{\small Per eseguire i logaritmi possiamo avvalerci della tabella dei generatori.} %TODO: magari approfondire e fare anche un esempio.

%TODO: [...] creando così la tabella degli algoritmi.

\subsubsection{Divisione}

\textsf{\small Si fa prendendo i logaritmi dei due numeri e eseguedo il modulo 255.}

\textsf{\small $g^{(x - y) mod 255}$ dove g è uno dei generatori. }

%\subsubsection{Inverso} %TODO: Multiplication Inverse/Inverso Moltiplicativo

%TODO: uncomment/remove?

%TODO: 
%\textsf{\small }

% ----------- SECTION: IL TEOREMA FONDAMENTALE DELLA TEORIA DI GALOIS  ----------------

\section{Il teorema fondamentale della teoria di Galois} %TODO: oppure non in una section? oppure proprio non metterlo?
\textsf{\small È un teorema che crea un legame tra gli intercampi di un'estensione di Galois e i sottogruppi del relativo gruppo di Galois.} %TODO: \cite{} Wikipedia

\textsf{\small Data un'estensione di campi $M/K$ con gruppo di Galois $G = \text{ Gal(M/K)}$ definiamo \emph{i} e \emph{j} (due funzioni): } %TODO: \cite{} Wikipedia

\begin{itemize}
	\item \textsf{\small Per ogni intercampo L (cioè tale che $K \subseteq L \subseteq M$) poniamo \emph{i}(L) = Gal(M/L), ossia il sottogruppo degli automorfismi di M che lasciano fissi gli elementi di L.}
	\item \textsf{\small Per ogni sottogruppo H di G, \emph{j}(H) che classicamente è indicato con $M^H$, è l'intercampo costituito dagli elementi di M lasciati fissi da tutti gli automorfismi di H.}
\end{itemize}

\textsf{\small } %TODO: \cite{} Wikipedia

% -------------------------------- FINE CAPITOLO --------------------------------------
	
	% ================================ L'IMPLEMENTAZIONE ====================================

\chapter{L'implementazione}

%TODO: Implementazione in C++
%TODO: Con gui e command line o solo gui.

%TODO: Volendo anche Java?

%TODO: Presenterò/Mostrerò due implementazioni una fatta specificamente per questo progetto in C++ e un'altra in Java che avevo fatto per altri progetti.
%TODO: Quella in C++ sarà più completa rispetto di quella in Java.

% =======================================================================================

% ---------------------------- SECTION: INTRODUZIONE ------------------------------------

\section{Introduzione}

\textsf{\small } %TODO: In questo capitolo, presenterò due implementazioni, una elaborata esclusivamente per questo progetto in C++ e un'altra in Java che avevo creato per altri progetti universitari. Quella in C++ risulterà più completa rispetto a quella in Java.

% ---------------------------- SECTION: IMPLEMENTAZIONE IN C++ --------------------------

\section{Implementazione in C++}

\textsf{\small } %TODO: Ho adottato C++23 per questo progetto. In esso sono presenti una interfaccia grafica e una applicazione da linea di comando, entrambe hanno le stesse operazioni.

%TODO: Inanzittutto, esibirò, le funzioni riguardanti la matematica di Galois di cui mi sono avvalso.

\subsection{Matematica di Galois}

\textsf{\small } %TODO: Riguardo alla matematica nel campo di Galois, ho adoperato tre funzioni: una che implementa l'addizione e la sottrazione e due per la moltiplicazione.

\textsf{\small Nel campo di Galois, sia l'addizione che la sottrazione sono semplicemente un'operazione di XOR. Questa funzione prende due parametri x ed y di tipo \emph{uint8\_t} (che corrisponde a un \emph{unsigned char}) e restituisce lo XOR tra essi.}

\begin{figure}[H]
	\centering
	\includegraphics[width=1\textwidth, height=1\textheight, keepaspectratio]{./images/code/cpp/galois_math/galois_addition_subtraction.PNG}
	\caption{Addizione e sottrazione nel campo di Galois}
	\label{fig:galois_addition_subtraction}
\end{figure}

\textsf{\small \emph{galois\_multiplication()} prende due \emph{uint8\_t} come parametri e restituisce la moltiplicazione tra questi nel campo di Galois.}

\textsf{\emph{static constexpr} indicano che la funzione può essere eseguita a compile time. \emph{noexcept} indica che la funziona non lancia eccezioni. \emph{[[nodiscard]]} indica che il risultato che viene restituito non può essere ignorato, ma deve essere utilizzato.}

%TODO: Aggiungere reference alla parte nel capitolo della Matematica o aggiungere breve spiegazione.

\textsf{\small Facciamo un loop su ogni bit del byte e verifichiamo se il secondo valore \emph{y} ha il bit meno significativo attivo (y \& 0x01) allora aggiungiamo \emph{x} (ovvero eseguiamo uno XOR) al risultato. Dopodiché verifichiamo se il bit più significativo \emph{high\_bit} è attivo in \emph{x}. Poi ruotiamo x di 1. Se l'\emph{high\_bit} è \emph{true} eseguiamo uno XOR tra x e il polinomio irriducibile \emph{IRREDUCIBLE\_POLYNOMIAL}, ovvero $0x1\text{B}$. Infine ruotiamo il secondo valore di 1 per ruotarlo a destra. Poi restituiamo il risultato finale delle operazioni.}

\begin{figure}[H]
	\centering
	\includegraphics[width=1\textwidth, height=1\textheight, keepaspectratio]{./images/code/cpp/galois_math/galois_multiplication.PNG}
	\caption{Moltiplicazione nel campo di Galois}
	\label{fig:galois_multiplication}
\end{figure}

\textsf{\small \emph{xtime(const uint8\_t\& x)} prende un numero in input e restituisce la round constant.} %TODO:

\textsf{\small Questa funzione è utilizzata per generare la round constant.}
\textsf{\small L'algoritmo di questa funzione è il seguente: }

\begin{itemize}
	\item \textsf{\small il round\_constant(1) = 1.}
	\item \textsf{\small round\_constant(i) = $2 \cdot \text{round\_constant}(i - 1) \text{se round\_constant(i - 1) < 0x80}$}
	\item \textsf{\small round\_constant($2 \cdot \text{round\_constant(i - 1)}$) $ \oplus \hspace{0.3mm} \text{0x11B } \ge 0x80$ }
\end{itemize}

\begin{figure}[H] %TODO: update.
	\centering
	\includegraphics[width=1\textwidth, height=1\textheight, keepaspectratio]{./images/code/cpp/galois_math/xtime.PNG}
	\caption{Generatrice delle costanti di round}
	\label{fig:xtime}
\end{figure}

%TODO: Come prima cosa mostrerò l'implementazione dell'algoritmo di AES con tutte le varie operazioni all'interno di ogni round di AES, ovvero: add round key, sub bytes, shift rows e mix columns. Dopodiché presenterò la parte di Key Expansion, ovvero come vengono ottenute le chiavi per ogni round, anch'esso composto da queste fasi: rot word, sub word, rcon.

%\subsection{Cifratura} %TODO: uncomment?

\subsection{Add Round Key}

\textsf{\small Nella fase di \emph{add\_round\_key()} la chiave di round viene aggiunta alla matrice di stato.}

\textsf{\small La funzione prende la matrice di stato 4x4 (formata da due std::array di grandezza BLOCK\_WORDS che indica il numero di words in un blocco, ovvero 4) e la chiave del round come puntatore e le aggiunge.}

\begin{figure}[H]
	\centering
	\includegraphics[width=1\textwidth, height=1\textheight, keepaspectratio]{./images/code/cpp/encryption/add_round_key.PNG}
	\caption{Add Round Key}
	\label{fig:add_round_key}
\end{figure}

\subsection{Sub Bytes}

\textsf{\small Nella funzione \emph{sub\_bytes} ogni byte della matrice di stato viene sostituito con quelli presenti nella S-BOX.}

\textsf{\small Quindi, nella funzione viene passata la matrice di stato come reference, quindi tutti le modifiche verranno applicate anche all'esterno della funzione e poi viene eseguito un loop e ogni elemento viene sostituito con il corrispettivo della S-BOX.}

\begin{figure}[H]
	\centering
	\includegraphics[width=1\textwidth, height=1\textheight, keepaspectratio]{./images/code/cpp/encryption/sub_bytes.PNG}
	\caption{Sub Bytes}
	\label{fig:sub_bytes}
\end{figure}

\subsection{Shift Rows}

\textsf{\small Nel passaggio di \emph{shift rows} le righe della matrice di stato verranno \emph{shiftate} di una posizione la seconda riga, di due posizione la terza e di tre la quarta.}

\textsf{\small Per fare questo ci avvaliamo di due funzioni, una \emph{shift\_row} per shiftare effettivamente le righe e nell'altra \emph{shift\_rows} per chiamare la precedente funzione per shiftare delle posizioni stabilite.}

%TODO: aggiungere altro?

\begin{figure}[H]
	\centering
	\includegraphics[width=1\textwidth, height=1\textheight, keepaspectratio]{./images/code/cpp/encryption/shift_row.PNG}
	\caption{Shift Row}
	\label{fig:shift_row}
\end{figure}

\textsf{\small } %TODO:

\begin{figure}[H]
	\centering
	\includegraphics[width=1\textwidth, height=1\textheight, keepaspectratio]{./images/code/cpp/encryption/shift_rows.PNG}
	\caption{Shift Rows}
	\label{fig:shift_rows}
\end{figure}

\subsection{Mix Columns}

\textsf{\small La procedura \emph{mix\_columns} prende in input la matrice di stato, mescola i suoi bytes.} %TODO:

%TODO: Aggiungere altro.

\begin{figure}[H]
	\centering
	\includegraphics[width=1\textwidth, height=1\textheight, keepaspectratio]{./images/code/cpp/encryption/mix_columns.PNG}
	\caption{Mix Columns}
	\label{fig:mix_columns}
\end{figure}

%TODO: funzioni di decifrazione?

\subsection{Key Expansion}

\textsf{\small In questa funzione vengono generate le altre chiavi dei rounds a partire dalla prima chiave. Gli viene passata un array con la chiave, una word e la tipologia di AES: 128, 192 o 256.} %TODO:

\textsf{\small Dopodiché eseguiamo le operazioni di: \emph{rot\_word}, \emph{sub\_word}, e \emph{rcon}. Dopodiché viene eseguito uno XOR tra la chiave e il rcon. Dopodiché si continua a eseguire uno XOR con le chiavi precedenti.}

%TODO: Aggiungere altro.

\begin{figure}[H]
	\centering
	\includegraphics[width=1\textwidth, height=1\textheight, keepaspectratio]{./images/code/cpp/key_expansion/key_expansion.PNG}
	\caption{Key Expansion}
	\label{fig:key_expansion_code}
\end{figure}

\subsubsection{Rot Word}

\begin{figure}[H]
	\centering
	\includegraphics[width=1\textwidth, height=1\textheight, keepaspectratio]{./images/code/cpp/key_expansion/rot_word.PNG}
	\caption{Rot Word}
	\label{fig:rot_word}
\end{figure}

\textsf{\small In questa operazione ogni byte (word di 32 bits, ovvero 4 bytes) viene ruotato di 1 posizione.} %TODO: Aggiungere altro.

\subsubsection{Sub Word}

\begin{figure}[H]
	\centering
	\includegraphics[width=1\textwidth, height=1\textheight, keepaspectratio]{./images/code/cpp/key_expansion/sub_word.PNG}
	\caption{Sub Word}
	\label{fig:sub_word}
\end{figure}

\textsf{\small Nella procedura \emph{sub\_word()} ogni byte della chiave viene sostituita con quella della S-BOX.} %TODO:

\subsubsection{Rcon}

\begin{figure}[H]
	\centering
	\includegraphics[width=1\textwidth, height=1\textheight, keepaspectratio]{./images/code/cpp/key_expansion/rcon.PNG}
	\caption{Rcon}
	\label{fig:rcon}
\end{figure}

\textsf{\small Nella funzione \emph{rcon}, le \emph{round constants} vengono generate attraverso una funzione ricorsiva.} %TODO:

\subsubsection{Xor Blocks} %TODO: uncomment?

\textsf{\small Con questa funzione eseguiamo uno XOR per ogni bit \emph{i} tra i blocchi x e y e assegniamo il risultato a ogni bit del blocco z. }

\textsf{\small Il loop viene eseguito in base alla grandezza del blocco.}

\begin{figure}[H]
	\centering
	\includegraphics[width=1\textwidth, height=1\textheight, keepaspectratio]{./images/code/cpp/key_expansion/xor_blocks.PNG}
	\caption{Xor Blocks}
	\label{fig:xor_blocks}
\end{figure}

%\textsf{\small } %TODO:

\subsection{Modes}

\textsf{\small } %TODO:

%TODO: subsection: Modes con codice? Questo si potrebbe fare.

\subsubsection{ECB}

\textsf{\small ECB è la modalità più semplice e anche quella che non dovrebbe mai essere usata.} %TODO: aggiungere reference alla spiegazione della modalità.

\textsf{\small In questa modalità, semplicemente, ogni blocco viene cifrato com'è. Nessun vettore di inizializzazione viene utilizzato. Lo stesso input genererà lo stesso identico output.}

\textsf{\small Questa modalità inoltre accetta solo blocchi divisibili per 16, questo viene garantito attraverso la funzione \emph{verify\_length()} che verifica e lancia un'eccezione altrimenti.}

\begin{figure}[H]
	\centering
	\includegraphics[width=1\textwidth, height=1\textheight, keepaspectratio]{./images/code/cpp/modes/encrypt_ECB.PNG}
	\caption{Cifratura ECB}
	\label{fig:encrypt_ECB}
\end{figure}

\textsf{\small Per la decifrazione è lo stesso procedimento, ma inverso. } %TODO:

\begin{figure}[H]
	\centering
	\includegraphics[width=1\textwidth, height=1\textheight, keepaspectratio]{./images/code/cpp/modes/decrypt_ECB.PNG}
	\caption{Decifrazione ECB}
	\label{fig:decrypt_ECB}
\end{figure}

\subsubsection{CBC}

\textsf{\small } %TODO:

\begin{figure}[H]
	\centering
	\includegraphics[width=1\textwidth, height=1\textheight, keepaspectratio]{./images/code/cpp/modes/encrypt_CBC.PNG}
	\caption{Cifratura CBC}
	\label{fig:encrypt_CBC}
\end{figure}

\textsf{\small } %TODO:

\begin{figure}[H]
	\centering
	\includegraphics[width=1\textwidth, height=1\textheight, keepaspectratio]{./images/code/cpp/modes/decrypt_CBC.PNG}
	\caption{Decifrazione CBC}
	\label{fig:decrypt_CBC}
\end{figure}

\subsubsection{CFB}

\textsf{\small } %TODO:

\begin{figure}[H]
	\centering
	\includegraphics[width=1\textwidth, height=1\textheight, keepaspectratio]{./images/code/cpp/modes/encrypt_CFB.PNG}
	\caption{Cifratura CFB}
	\label{fig:encrypt_CFB}
\end{figure}

\textsf{\small }

\begin{figure}[H]
	\centering
	\includegraphics[width=1\textwidth, height=1\textheight, keepaspectratio]{./images/code/cpp/modes/decrypt_CFB.PNG}
	\caption{Decifrazione CFB}
	\label{fig:decrypt_CFB}
\end{figure}

\subsection{Paddings}

\textsf{\small } %TODO:

\begin{figure}[H]
	\centering
	\includegraphics[width=1\textwidth, height=1\textheight, keepaspectratio]{./images/code/cpp/padding/add_padding0.PNG}
	\caption{Aggiunta del padding (1/2)}
	\label{fig:add_padding0}
\end{figure}

\begin{figure}[H]
	\centering
	\includegraphics[width=1\textwidth, height=1\textheight, keepaspectratio]{./images/code/cpp/padding/add_padding1.PNG}
	\caption{Aggiunta del padding (2/2)}
	\label{fig:add_padding1}
\end{figure}

\textsf{\small }

\begin{figure}[H]
	\centering
	\includegraphics[width=1\textwidth, height=1\textheight, keepaspectratio]{./images/code/cpp/padding/remove_padding0.PNG}
	\caption{Rimozione del padding (1/2)}
	\label{fig:remove_padding0}
\end{figure}

\begin{figure}[H]
	\centering
	\includegraphics[width=1\textwidth, height=1\textheight, keepaspectratio]{./images/code/cpp/padding/remove_padding1.PNG}
	\caption{Rimozione del padding (2/2)}
	\label{fig:remove_padding1}
\end{figure}

\textsf{\small } %TODO:

\subsection{API}

\textsf{\small }

%TODO: solo encrypt e decrypt?

\subsubsection{Cifratura}

\textsf{\small }

\begin{figure}[H]
	\centering
	\includegraphics[width=1\textwidth, height=1\textheight, keepaspectratio]{./images/code/cpp/api/encrypt.PNG}
	\caption{Cifratura}
	\label{fig:encrypt}
\end{figure}

\subsubsection{Decifratura}

\textsf{\small }

\begin{figure}[H]
	\centering
	\includegraphics[width=1\textwidth, height=1\textheight, keepaspectratio]{./images/code/cpp/api/decrypt.PNG}
	\caption{Decifratura}
	\label{fig:decrypt}
\end{figure}

\textsf{\small } %TODO:

% ---------------------------- SECTION: IMPLEMENTAZIONE IN JAVA -------------------------

\section{Implementazione in Java}

\textsf{\small } %TODO: 

\begin{figure}[H]
	\centering
	\includegraphics[width=1\textwidth, height=1\textheight, keepaspectratio]{./images/code/java/constructor_and_member_variables.PNG}
	\caption{Costruttore e variabili membre}
	\label{fig:constructor_and_member_variables}
\end{figure}

\textsf{\small } %TODO: 

\begin{figure}[H]
	\centering
	\includegraphics[width=1\textwidth, height=1\textheight, keepaspectratio]{./images/code/java/nonce_getRandomBytes.PNG}
	\caption{Nonce}
	\label{fig:nonce_getRandomBytes}
\end{figure}

\textsf{\small } %TODO: 

\begin{figure}[H]
	\centering
	\includegraphics[width=1\textwidth, height=1\textheight, keepaspectratio]{./images/code/java/getKeyFromPassword.PNG}
	\caption{Password}
	\label{fig:getKeyFromPassword}
\end{figure}

\textsf{\small } %TODO: 

\begin{figure}[H]
	\centering
	\includegraphics[width=1\textwidth, height=1\textheight, keepaspectratio]{./images/code/java/encrypt.PNG}
	\caption{Cifratura}
	\label{fig:encrypt_java}
\end{figure}

\textsf{\small } %TODO: 

\begin{figure}[H]
	\centering
	\includegraphics[width=1\textwidth, height=1\textheight, keepaspectratio]{./images/code/java/decrypt.PNG}
	\caption{Decifratura}
	\label{fig:decrypt_java}
\end{figure}

\textsf{\small } %TODO: 

\begin{figure}[H]
	\centering
	\includegraphics[width=1\textwidth, height=1\textheight, keepaspectratio]{./images/code/java/encryptFile.PNG}
	\caption{Funziona per la cifratura di un File}
	\label{fig:encryptFile}
\end{figure}

\textsf{\small } %TODO: 

\begin{figure}[H]
	\centering
	\includegraphics[width=1\textwidth, height=1\textheight, keepaspectratio]{./images/code/java/decryptFile.PNG}
	\caption{Funziona per la decifrazione di un File}
	\label{fig:decryptFile}
\end{figure}

\textsf{\small } %TODO: 

% -------------------------------- FINE CAPITOLO ----------------------------------------
	
	% ================================ LE MODALITÀ DI AES ===================================

\chapter{Le modalità di AES}

%TODO: ECB, CBC, CTR, GCM, CBC-MAC, CFB, OCB, XTS, OFB
%TODO: Section sulle MAC
%TODO: Section Authenticated encryption
%TODO: Consigli generali? Non riusare mai lo stesso IV (spiegare cos'è la IV) e rendere l'IV non predicibile!
%TODO: spiegare cos'è il nonce (cercare traduzione in italiano).
%TODO: IV vs nonce.

%TODO: aggiungere delle immagini se possibile.

%TODO: Aggiungere i vari tipi di padding? Prima o dopo la spiegazione delle modalità? Prima direi, dopo l'iv, nonce, ecc.

% =======================================================================================

% ---------------------------- SECTION: INTRODUZIONE ------------------------------------

%TODO: In questo capitolo/In questa parte tratteremo le varie modalità di crittografia disponibili su/per AES. Queste ci permetteranno di cifrare messaggi di varia lunghezza.

\section{Introduzione}

% ---------------------------- SECTION: PERCHÉ SERVE UNA/LE MODALITÀ? PERCHÉ USARE UNA/LE MODALITÀ? A COSA SERVONO LE MODALITÀ? ------------

\section{A cosa servono le modalità?} %TODO: A cosa servono le modalità? oppure Perché usare le modalità

%TODO: Le modalità sono necessarie, perché AES è un cifrario a blocchi, prende in input un blocco e una chiave di 16 bytes

%TODO: Le modalità servono, perché AES è un cifrario a blocchi di 16 bytes, dopo aver preso un blocco e una chiave in input restituisce un altro blocco della medesima grandezza/lunghezza.

%TODO: Quindi, se si desidera/vuole cifrare una sequenza di bytes di lunghezza/grandezza maggiore di 16 bytes, bisogna/è necessario utilizzare una \emph{modalità di operazione} o \emph{modalità di cifratura}.

%TODO: Queste modalità sono indipendenti dal tipo di cifrario sottostante/che sta alla base.

\textsf{\small }

% -------------------------- SECTION: IV, NONCE, SALT E PEPPER --------------------------

\section{IV, Nonce, Salt e Pepper} %TODO: IV, Nonce, Sale e Pepe?

%TODO: Prima di poter trattare le modalità è necessario introdurre alcuni componenti/elementi che ci serviranno (nella creazione ...). (Questi sono l'IV, il nonce, il sale e il pepe).

\textsf{\small }

\subsection{Nonce | Number Used Once} %TODO: oppure solo Nonce

%TODO: Aggiungere dove viene usato.

%TODO: Il nonce, \emph{\textbf{n}umber used \textbf{once}} (numero utilizzato una volta), sono dei bits (ovvero un numero) che vengono utilizzati una singola volta. Il riutilizzo non è proibito, ma deve essere limitato.

%TODO: Viene utilizzato nei protocolli di autenticazione per garantire la sicurezza nelle comunicazioni (private) (ed evitare replay attacks [da spiegare in un altro capitolo (e aggiungere la \ref alla pagina volendo)]).

\textsf{\small }

\subsubsection{Nonce sequenziali}

%TODO: Prevenisco/Garantiscono la non ripetizione, se molto grandi.

\textsf{\small }

%TODO: subsubsection Nonce random?

\subsection{IV | Initialization Vector} %TODO: Oppure solo IV; prima era L'IV; Initialization Vector | IV

%TODO: L'initialization vector o anche definito/chiamato starting variables è un input di dimensione fissa, ovvero un numero arbitrario che viene fornito/adoperato per impostare lo stato iniziale di un algoritmo crittografico. Viene utilizzato in varie modalità di cifratura per randomizzare la cifratura in modo da produrre/generare diversi testi cifrati anche se lo stesso testo viene cifrato molteplici volte.

%TODO: È simile al nonce, ma deve essere casuale/random. Quindi i nonce sequenziali non andrebbero bene.

%TODO: È prioritario/molto importante che l'IV sia:
%TODO: itemize: impredicibile
%TODO: sempre diverso, mai riutilizzare lo stesso/un IV!

\textsf{\small }

\subsection{IV vs Nonce} %TODO: forse questa non serve.

\textsf{\small }

\subsection{Salt} %TODO: Oppure Sale; Oppure Salt | Password Salting

%TODO: Il salt/sale/password salting viene utilizzato nelle password per incrementarne la sicurezza. Vengono aggiunti dei caratteri unici casuali alla password, in questo modo un attaccante non necessita solamente di un dizionario di password comuni, ma anche di uno per ogni tipo di sale possibile.

%TODO: Questo processo/tecnica è molto importante e viene utilizzata nei database affinché le password degli utenti siano sicure. (e non possano essere rubate).
%TODO: Le password non vengono mai, possibilmente, memorizzate nei database in chiaro, ma gli viene aggiunto il salt e poi viene eseguito l'hashing su esse.

\textsf{\small }

\subsection{Pepper} %TODO: Oppure Pepe

%TODO: Il pepper/pepe è simile al salt/sale, ma questo a differenza del sale che è appena/meramente unico più che segreto, è un (singolo) carattere segreto appeso/aggiunto alla fine della password.

%TODO: Questo, (a differenza del sale) non viene memorizzato nello stesso database assieme alla password hash e al sale, ma a parte.

\textsf{\small }

%TODO: subsection: IV, Nonce, Salt, Pepper o magari salt e pepper non servono.

%TODO: subsection IV vs Nonce

% ---------------------------- SECTION: LE MODALITÀ -------------------------------------

\section{Le modalità} %TODO: oppure trovare un altro titolo: Panoramica sulle modalità? però l'ho già usato come titolo "panoramica".

%TODO: Qui, di seguito elencherò le caratteristiche e proprietà delle principali modalità di cifratura:

\textsf{\small }

%TODO: encryption vs message integrity
%TODO: volendo anche il MAC qui e poi anche dopo
%TODO: MAC, meccanismo a tre componenti: Secret Key, MAC Signing Algorithm, MAC Validation Algorithm.

\subsection{Modalità di cifratura senza integrità del messaggio}

%TODO: Queste modalità forniscono la cifratura del messaggio, ma non garantisco l'integrità del messaggio, ovvero/quindi non è assicurato che il messaggio non sia stato manomesso/alterata e quindi non è possibile accertare l'autenticazione del messaggio originale. %TODO: magari riguardare la parte dell'"accertare l'autenticazione del messaggio originale".

\textsf{\small }

\subsubsection{ECB | Electronic Code Book}

%TODO: Questa modalità di cifratura divide il nostro messaggio in diversi blocchi e ognuno di questi viene cifrato separatamente.

%TODO: Questa modalità manca il principio di diffusione (aggiungere \ref) e quindi cifra lo stesso messaggio nello stesso testo cifrato, quindi non vela (il pattern) dei dati/i dati.

%TODO: Quindi, se si encripta/cifra più di un blocco con la stessa chiave, non andrebbe usato.
%TODO: In generale, ECB è considerata una modalità obsoleta, da non utilizzare. È una modalità molto debole e poco sicura.

\textsf{\small }

\begin{figure}[H]
	\centering
	\includegraphics[width=.9\textwidth, height=.9\textheight, keepaspectratio]{./images/aes_modes/ecb_encryption.png} % mettere 1 e 1?
	\caption{ECB}
	\label{fig:ecb}
\end{figure}

\subsubsection{CBC | Cipher Block Chain}

%TODO: La modalità CBC cerca di ovviare ai problemi dell'ECB aggiungendo della casualità a ogni operazione di cifratura usufruendo di un initialization vector (IV) per il primo blocco e dopodichè applicando uno XOR a ogni blocco di testo in chiaro con il blocco cifrato precedente/precedentemente.

\textsf{\small }

\begin{figure}[H]
	\centering
	\includegraphics[width=1\textwidth, height=1\textheight, keepaspectratio]{./images/aes_modes/cbc.png} % mettere 1 e 1?
	\caption{CBC}
	\label{fig:cbc}
\end{figure}

%TODO: Questa modalità è sicura come uno schema probabilistico e la confidenzialità non viene raggiunta se l'IV è un nonce semplice. %TODO: forse approfondire cos'è un nonce semplice

\subsubsection{CFB | Cipher Feedback}

%TODO: 

\textsf{\small }

\subsubsection{OCB | Offset Codebook} %TODO: section MACS

%TODO: 

\textsf{\small }

\subsubsection{CTR | Counter Mode}

%TODO: 

\textsf{\small }

\subsubsection{XTS | AES-XTS (XEX) Tweakable Blk Cipher}

%TODO: 

\textsf{\small }

%TODO: subsection Block cipher modes, encryption but not message integrity
%TODO: subsection ECB | Electronic Code Book
%TODO: subsection Cipher Block Chain | CBC
%TODO: subsection CFB | Cipher Feedback
%TODO: subsection OCB | Offset Codebook (questo va nella subsection MACS)
%TODO: subsection CTR | Counter Mode
%TODO: subsection XTS | AES-XTS (XEX) Tweakable Blk Cipher 

%TODO: subsection EAX | encrypt-then-authenticate-then-translate. metterlo o no?

\subsection{MACS | Message Authentication Codes}

%TODO: 

\textsf{\small }

\subsubsection{ALG1-6}

%TODO: 

\textsf{\small }

\subsubsection{CMAC | Cipher-based Message Authentication Code}

%TODO: 

\textsf{\small }

\subsubsection{HMAC | Keyed-hash Message Authentication Code}

%TODO: 

\textsf{\small }

\subsubsection{GMAC | Galois Message Authentication Code}

%TODO: 

\textsf{\small }

\subsubsection{CBC-MAC}

%TODO: 

\textsf{\small }

%TODO: section/subsection MACS
%TODO: spiegare cos'è il MAC e la filosofia dietro.
%TODO: subsection ALG1-6, CMAC (Cipher-based Message Authentication Code), HMAC (keyed-hash message authentication code o hash-based message authentication code), GMAC (Galois Message Authentication Code), CBC-MAC (CBC-MAC qui o sopra?)

\subsection{AEAD | Authenticated Encryption with Associated Data}

%TODO: 

\textsf{\small }

\subsubsection{CCM | Cipher Block Chaining}

%TODO: 

\textsf{\small }

\subsubsection{GCM | Galois Counter Mode}

%TODO: 

\textsf{\small }

%TODO: section/subsection Authenticated Encryption (AEAD (Authenticated Encryption with Associated Data))
%TODO: CCM | Cipher Block Chaining e GCM | Galois Counter Mode
	
	%TODO: Prima dell'implementazione facciamo: "Le modalità di AES" oppure anche prima del capitolo "La matematica dietro AES" oppure anche no!
	
	%TODO: Implementazione %TODO: oppure dipende dall'implementazione, se nell'implementazione c'è solo AES, allora la parte delle Modalità la metto dopo, altrimenti prima.
	
	%TODO: Pros e Cons (Tipi Attacchi su AES, vulnerabilità?, ecc.) %TODO: volendo resistance to linear and differential cryptoanalysis, wide tail strategy, impractical related key attacks.
	
	%TODO: Conclusioni
	
	%TODO: Bibliografia e Indice Analitico.
	
	\backmatter
	
\end{document}

% ----------------------------- END DOCUMENT -----------------------------------------