% ================================ STORIA DI AES ======================================

\chapter{Storia di AES}

% ======================================================================================

%TODO: stream cipher vs block cipher \section
%TODO: kerchoffs laws \section.

% ---------------------------- SECTION: INTRODUZIONE ----------------------------------

%\newpage

\section{Introduzione}

%TODO: \textsf o \textsc o altro ?

%TODO: a blocchi simmetrico o a blocchi simmetrici?
\textsf{\small \textbf{AES} (\emph{\textbf{A}dvanced \textbf{E}ncryption \textbf{S}tandard}) è un cifrario a blocchi simmetrico, inventato da due matematici belghi, Vincent Rijmen e Joan Daemen, da cui viene il nome \emph{Rijndael}, nel 1998 per sostituire il precedente standard: DES (\emph{\textbf{D}ata \textbf{E}ncryption \textbf{S}tandard}).}

%TODO: aggiungere la differenza tra AES e Rijndael, sono diversi.
%TODO: Rijndael: gestisce blocchi e chiavi di differenti dimensioni. (blocco e chiave di 32 bit con 128 bit come minimo e 256 come massimo).
%TODO: AES: gestisce un blocco di dimensione fissa (128 bit) e la chiave può essere da 128, 192 o 256.

% ---------------------------- SECTION: STORIA DI AES ---------------------------------

%\newpage %TODO: questo newpage da problema, è da fixare

\section{Breve storia di AES} 

%TODO: Partire dal DES, dai problemi del DES e del triplo DES
%TODO: Nist 1997
%TODO: I concorrenti di AES: serpent, twofish, ecc. e dire per quale motivo è stato scelto Rijndael come AES alla fine.

\textsf{\small DES era divenuto lo standard dopo un bando dell'NBS (\emph{\textbf{N}ational \textbf{B}ureau of \textbf{S}tandards}), oggi NIST \emph{(\textbf{N}ational \textbf{I}nstitute for \textbf{S}ecurity and \textbf{T}echnology)} per trovare un buon e sicuro algoritmo per proteggere le comunicazioni private dei cittadini americani.}

%TODO: "rettificandone" o "alterandone"
\textsf{\small Venne così proposto un algoritmo chiamato \emph{Lucifer}, sviluppato dall'\emph{IBM} che dopo esser stato modificato dall'NSA (\emph{\textbf{N}ational \textbf{S}ecurity \textbf{A}gency}), riducendone la grandezza della chiave da 128 a 56 bits e rettificandone le funzioni contenute nell'S-box, venne designato come \emph{Data Encryption Standard} (\textbf{DES}).}

%TODO: "in lungo e in largo" magari è poco raffinata come espressione.
%TODO: questa è da rivedere! Mi sembra abbia poco senso compiuto
\textsf{\small DES regnò per 20 anni, venne studiato in lungo e in largo dagli accademici e criptoanalisti di tutto il mondo, grazie a ciò, ci fu finalmente per la prima volta un cifrario certificato che tutti potevano studiare: nacque così il moderno campo della crittografia.}

\textsf{\small Negli anni, molti sfidarono DES e dopo diverse battaglie fu finalmente sconfitto.} %TODO: ampliare come venne sconfitto?

%TODO: "fermare" o "ovviare"
%TODO: magari spiegare perché era lento più nel dettaglio. (perché combinevamo il des tre volte)
\textsf{\small L'unico modo per ovviare a questi attacchi era quello di combinare des tre volte, formando il 3DES (\emph{Triplo DES}). Il problema di questo però era la sua lentezza.} %TODO: estremamente lento


%TODO: "indire" è il verbo corretto?
\textsf{\small Per questo, nel 1997, il NIST indisse un nuovo bando per cercare un nuovo algoritmo di cifratura, forte come il triplo-DES, ma veloce e flessibile.}

\begin{figure}[H]
	\centering
	\includegraphics[width=1\textwidth, height=1\textheight, keepaspectratio]{./images/des_vs_aes/aes-vs-des.png}
	\caption{AES vs DES}
	\label{fig:aes_vs_des}
\end{figure}

%TODO: "spuntò" o "emerse" ?
%TODO: "semplicità"? Approfondire un po' di più magari
\textsf{\small Vari algoritmi competerono: Serpent, Twofish, MARS, RC6, ma alla fine spuntò Rijndael per la sua semplicità e velocità.} %TODO: magari aggiungere i link a pagine su questi algoritmi? con il footnote anche?

% ---------------------------- SECTION: AES VS RIJNDEAL ------------------------------

%\newpage

\section{AES vs Rijndeal}

\textsf{\small AES è un'implentazione di Rijndael, divenuto l'algoritmo di cifratura standard del governo degli Stati Uniti d'America.}
\textsf{\small Una differenza tra i due è che AES utilizza blocchi di dati da 128 bits, mentre Rijndael permette oltre a blocchi da 128, anche blocchi da 192 e 256 bits.} %TODO: "blocchi di dati"?

\textsf{\small Sia AES che Rijndael permettono una grandezza della chiave di 128, 192 o 256 bits, da cui ne ricaviamo il numero di rounds: 10, 12 o 14 rispettivamente.}

\begin{comment}
\begin{figure}[H]
	\centering
	\includegraphics[width=1\textwidth, height=1\textheight, keepaspectratio]{./images/.png}
	\caption{}
	\label{fig: }
\end{figure}
\end{comment}

%TODO: immagine AES vs Rijndael

% ------------------------ SECTION: SYMMETRIC VS ASYMMETRIC? --------------------------

%\newpage

\section{Cifratura simmetrica vs asimmetrica} %rinominare in "Cifrari simmetrici vs asimettrici"?

\textsf{\small Nella cifratura simmetrica viene usata una chiave sia per la cifratura che per la decifratura di un messaggio.}

\textsf{\small La cifratura asimmetrica è basata sul concetto di chiave pubblica e chiave privata. Vengono, quindi usate due chiavi sia per la cifratura che per la decifratura. Usiamo la chiave pubblica per cifrare il messagio e la chiave privata per decifrarlo.} %TODO: qui sto facendo un po' di controsenso.

\textsf{\small Ulteriori differenze: }

\begin{tabular}{|c|c|}
	\hline
	\textbf{Simmetrico} & \textbf{Asimmetrico} \\
	\hline
	\textsf{\small Richiede una sola  } & \textsf{\small Richiede due chiavi, } \\
	\textsf{\small chiave sia} & \textsf{\small  una pubblica e una privata,} \\
	\textsf{\small per la cifratura } & \textsf{\small una per cifrare e } \\
	\textsf{\small che la decifratura.} & \textsf{\small una per decifrare.} \\ %TODO: una per cifrare e una per decifrare?
	\hline
	\textsf{\small Lo spazio del testo cifrato è  } & \textsf{\small Lo spazio del testo cifrato è  } \\
	\textsf{\small lo stesso o più piccolo } & \textsf{\small lo stesso o più grande } \\
	\textsf{\small del messaggio originale.} & \textsf{\small del messaggio originale.} \\
	\hline
	\textsf{\small Il processo di cifratura } & \textsf{\small Il processo di cifratura } \\
	\textsf{\small è molto veloce.} & \textsf{\small è molto lento.} \\
	\hline
	\textsf{\small È usato quando un  } & \textsf{\small È usato per trasferire } \\
	\textsf{\small grosso ammontare di dati} & \textsf{\small piccole quantità} \\
	\textsf{\small deve essere trasferito.} & \textsf{\small  di dati.} \\
	\hline
	\textsf{\small Fornisce solamente } & \textsf{\small Fornisce confidenzialità, } \\ %TODO: non ripudio
	\textsf{\small la confidenzialità.} & \textsf{\small autenticità e non ripudio.} \\ %TODO: approfondire non ripudio?
	\hline
	\textsf{\small La chiave usata è di solito } & \textsf{\small La lunghezza della chiave } \\
	\textsf{\small di lunghezza 128 o 256 bits.} & \textsf{\small è di 2048 o più bits.} \\
	\hline
	\textsf{\small L'utilizzo delle risorse è basso.} & \textsf{\small L'utilizzo di risorse è alto.} \\
	\hline
	\textsf{\small Esempi: DES, 3DES, } & \textsf{\small Esempi: DSA, RSA, } \\
	\textsf{\small AES, RC4} & \textsf{\small Diffie-Hellman, ECC, El Gamal} \\
	%\textsf{\small } & \textsf{\small } \\
	\hline
\end{tabular}

\begin{figure}[H]
	\centering
	\includegraphics[width=1\textwidth, height=1\textheight, keepaspectratio]{./images/types_of_encryptions/types_of_encryption.png}
	\caption{Tipi di cifratura}
	\label{fig:encryption_types}
\end{figure}

\textsf{\small AES è di tipo simmetrico, quindi useremo la stessa chiave sia per criptare il nostro messaggio sia per decriptarlo. }

\begin{figure}[H]
	\centering
	\includegraphics[width=1\textwidth, height=1\textheight, keepaspectratio]{./images/types_of_encryptions/symmetric_vs_asymmetric/symmetric-key-what-is-cryptography.png}
	\caption{Cifratura a chiave simmetrica}
	\label{fig:symmetric_encryption}
\end{figure}

% -------------------------------- FINE CAPITOLO --------------------------------------