% ================================ LE MODALITÀ DI AES ===================================

\chapter{Le modalità di AES}

%TODO: ECB, CBC, CTR, GCM, CBC-MAC, CFB, OCB, XTS, OFB
%TODO: Section sulle MAC
%TODO: Section Authenticated encryption
%TODO: Consigli generali? Non riusare mai lo stesso IV (spiegare cos'è la IV) e rendere l'IV non predicibile!
%TODO: spiegare cos'è il nonce (cercare traduzione in italiano).
%TODO: IV vs nonce.

%TODO: aggiungere delle immagini se possibile.

% =======================================================================================

% ---------------------------- SECTION: INTRODUZIONE ------------------------------------

%TODO: In questo capitolo/In questa parte tratteremo le varie modalità di crittografia disponibili su/per AES. Queste ci permetteranno di cifrare messaggi di varia lunghezza.

\section{Introduzione}

% ---------------------------- SECTION: PERCHÉ SERVE UNA/LE MODALITÀ? PERCHÉ USARE UNA/LE MODALITÀ? A COSA SERVONO LE MODALITÀ? ------------

\section{A cosa servono le modalità?} %TODO: A cosa servono le modalità? oppure Perché usare le modalità

%TODO: Le modalità sono necessarie, perché AES è un cifrario a blocchi, prende in input un blocco e una chiave di 16 bytes

%TODO: Le modalità servono, perché AES è un cifrario a blocchi di 16 bytes, dopo aver preso un blocco e una chiave in input restituisce un altro blocco della medesima grandezza/lunghezza.

%TODO: Quindi, se si desidera/vuole cifrare una sequenza di bytes di lunghezza/grandezza maggiore di 16 bytes, bisogna/è necessario utilizzare una \emph{modalità di operazione} o \emph{modalità di cifratura}.

%TODO: Queste modalità sono indipendenti dal tipo di cifrario sottostante/che sta alla base.

\textsf{\small }

% -------------------------- SECTION: IV, NONCE, SALT E PEPPER --------------------------

\section{IV, Nonce, Salt e Pepper} %TODO: IV, Nonce, Sale e Pepe?

%TODO: Prima di poter trattare le modalità è necessario introdurre alcuni componenti/elementi che ci serviranno (nella creazione ...).

\textsf{\small }

\subsection{IV | Initialization Vector} %TODO: Oppure solo IV; prima era L'IV; Initialization Vector | IV

%TODO: 

\textsf{\small }

\subsection{Nonce | Number Used Once} %TODO: oppure solo Nonce

%TODO: 

\textsf{\small }

\subsection{Salt} %TODO: Oppure Sale; Oppure Salt | Password Salting

%TODO: 

\textsf{\small }

\subsection{Pepper} %TODO: Oppure Pepe

%TODO: 

\textsf{\small }

%TODO: subsection: IV, Nonce, Salt, Pepper o magari salt e pepper non servono.

%TODO: subsection IV vs Nonce

% ---------------------------- SECTION: LE MODALITÀ -------------------------------------

\section{Le modalità} %TODO: oppure trovare un altro titolo: Panoramica sulle modalità? però l'ho già usato come titolo "panoramica".

\textsf{\small }

%TODO: encryption vs message integrity
%TODO: volendo anche il MAC qui e poi anche dopo
%TODO: MAC, meccanismo a tre componenti: Secret Key, MAC Signing Algorithm, MAC Validation Algorithm.

%TODO: subsection Block cipher modes, encryption but not message integrity
%TODO: subsection ECB | Electronic Code Book
%TODO: subsection Cipher Block Chain | CBC
%TODO: subsection CFB | Cipher Feedback
%TODO: subsection OCB | Offset Codebook (questo va nella subsection MACS)
%TODO: subsection CTR | Counter Mode
%TODO: subsection XTS | AES-XTS (XEX) Tweakable Blk Cipher 

%TODO: subsection EAX | encrypt-then-authenticate-then-translate. metterlo o no?

%TODO: section/subsection MACS
%TODO: spiegare cos'è il MAC e la filosofia dietro.
%TODO: subsection ALG1-6, CMAC (Cipher-based Message Authentication Code), HMAC (keyed-hash message authentication code o hash-based message authentication code), GMAC (Galois Message Authentication Code), CBC-MAC (CBC-MAC qui o sopra?)

%TODO: section/subsection Authenticated Encryption (AEAD (Authenticated Encryption with Associated Data))
%TODO: CCM | Cipher Block Chaining e GCM | Galois Counter Mode