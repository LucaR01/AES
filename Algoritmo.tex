% ================================ L'ALGORITMO =========================================

\chapter{L'Algoritmo} %TODO: Rinominare in "Una panoramica sull'Algoritmo" o in realtà quello è per una sezione?

%TODO: Le varie sezioni: sub bytes, Shift Rows, mix columns, Add Round Key
%TODO: aggiungere immagini

%TODO: perché l'ultimo round è differente?

%TODO: 1. KeyExpansion (Rijndael Key Scheduler/AES Key Scheduler)
%TODO: 1. AddRoundKey (con lo XOR)

%TODO: 1. SubBytes (S-Box)
%TODO: 2. ShiftRows (1a riga niente, 2a riga = shift a sinistra di 1, 3a riga = shift di 2, 4a riga = shift di 3) (lo shift è come una ROL in x86)
%TODO: 3. MixColumns (Rijndael Key Scheduler/AES Key Scheduler)
%TODO: 4. AddRoundKey (con lo XOR)

%TODO: ShiftRows e MixColumns per la "Confusione e Diffusione" (teoria di Shannon).

%TODO: 10 rounds (fasi/cicli) per chiavi di 128-bit
%TODO: 12 rounds (fasi/cicli) per chiavi di 192-bit
%TODO: 14 rounds (fasi/cicli) per chiavi di 256-bit

% ======================================================================================

% ---------------------------- SECTION: INTRODUZIONE -----------------------------------

%\newpage

\section{Introduzione}

%In questo capitolo, tratteremo il funzionamento dell'algoritmo di AES con una panoramica dall'alto, per poi affrontare nel prossimo capitolo, più in dettaglio, la matematica dietro a questo. %TODO: "dietro a questo" suona male, meglio trovare qualcos'altro.

%TODO: In questo capitolo affronteremo il funzionamento dell'algoritmo di AES più nel dettaglio/con una visione dall'alto, per poi trattare nel prossimo capitolo la matematica dietro all'\emph{Advanced Encryption Standard}.

% ---------------------------- SECTION:  ----------------------------------

\newpage

\section{} %TODO: Le tre idee/concetti/principi dietro la Crittografia

%TODO: Alla base della crittografia, ci sono due importanti proprietà/principi dei cifrari a chiave simmetrica, elaborati dal padre della teoria dell'informazione, Claude Elwood Shannon, ovvero: \emph{diffusione} e \emph{confusione}.

%TODO: Il principio/proprietà della \emph{confusione} nasconde/vela la connessione tra il messaggio originale e il messaggio/testo cifrato. [per esempio cifrario di cesare (con annessa immagine volendo)]

%TODO: Il principio/proprietà di \emph{diffusione}, invece, riguarda il riordino/ridistribuzione/scombussolamento della posizione delle lettere/caratteri nel/del messaggio/testo. [un semplice esempio potrebbe essere la trasposizione delle colonne in una matrice oppure non lo scrivo.]

%TODO: Un altro concetto/astrazione? molto importante è quello della \emph{segretezza nella chiave}, ovvero che l'algoritmo alla base del cifrario è conosciuto, è pubblico, ma la sola conoscenza di questo non basta/non è sufficiente per poter ottenere/attingere/conseguire l'accesso alle informazioni, perché per poterlo raggiungere/attingere/acquisire/conseguire avraì bisogno/sarà necessario conoscere la chiave segreta. [volendo aggiungere altro sulla teoria dell'informazione, su Shannon, sull'importanza, ecc. Magari aggiungere anche un'immagine.]

%TODO: [aggiungere una o più immagini sulla teoria di Shannon]

% ---------------------------- SECTION:  ----------------------------------

\section{} %TODO: Una panoramica dell'Algoritmo/Una panoramica sull'Algoritmo (meglio una panoramica sull'algoritmo come nome!) oppure "L'Algoritmo in breve" o "Semplice Overview dell'Algoritmo".

%TODO: I dati di input vengono caricati in una matrice 4x4, anche chiamata \emph{state matrix} (matrice di stato), ogni cella contiene/rappresenta 1 byte di informazione, su questa compiamo diverse operazioni: sub-bytes (sostituzione dei bytes), shift rows (spostamento delle righe), mix columns (mescolamento delle colonne), add round key (aggiunta della chiave del round) per un numero di volte/rounds pari/in base alla grandezza della chiave. [immagine key size / Rounds]

%TODO:  Nel primo round svolgiamo uno XOR (bit-a-bit?) tra il messaggio/testo d'input e la chiave segreta. [inserire immagine tabella verità XOR]

%TODO: Lo XOR (EXclusive-OR) bit-a-bit è un'operazione di mascheratura dei bit, dove se i due bit di input sono diversi, allora genererà/produrrà un 1 in uscita, altrimenti se sono uguali, uno zero.

%TODO: Sezione/Subsection: Perché lo XOR è usato in crittografia?
%TODO: itemize?
%TODO: Lo XOR non fa fuoriuscire/fuori esce/leak informazioni (sull'input inziiale/originale?)
%TODO: Lo XOR è \emph{involutory function}, se lo applichi due volte, riottieni il testo originale.
%TODO: L'output dello XOR dipende da entrambi gli input. Non è così per le altre operazioni. (Scrivere: AND, OR, NOT, ecc.?)

%TODO: piccolo siparietto per tornare sull'argomento della sezione. \fleuron o \ornament

%TODO: Fare una subsection?: titolo: "Come vengono ottenute le chiavi per ogni round?" oppure no?
%TODO: Per poter eseguire/elaborare i rounds, l'algoritmo ha bisogno di molte chiavi [una per round?], queste vengono tutte derivate dalla chiave iniziale. (volendo scrivere: Questo è uno dei motivi per cui AES viene criticato, perché tutte le chiavi vengono derivate dalla chiave originale).

%TODO: Il procedimento è questo: [mettere delle immagini a riguardo]

%TODO: enumerate: 
%TODO: 1) Sposta la prima cella dell'ultima colonna della precedente chiave, in fondo/basso alla colonna.
%TODO: 2) Ogni byte viene eseguito in una substituion box [che lo mapperà in qualcos' altro].
%TODO: 3) Poi viene effettuato uno XOR tra la colonna e un \emph{round costant} (una costante di round) che è diversa per ogni round.
%TODO: 4) Infine viene realizzato uno XOR con la prima colonna della precedente chiave.

%TODO: Per quanto riguarda le altre colonne, vengono semplicemente eseguiti degli XOR con la stessa colonna della precedente chiave. [magari spiegare meglio] [magari mettere un'immagine a riguardo] [eccetto per le chiavi a 256 bit che sono un po' più complicate] [approfondire il perché siano più complicate]

%TODO: Dopo aver ottenuto queste chiavi vengono compiuti i vari rounds.

%TODO: Per ogni round, eseguiamo questi passaggi tranne per l'ultimo dove nonn effettuiamo il passaggio delle "Mix columns", perché non aumenterebbe la sicurezza e semplicemente rallenterebbe: [magari approfondire il perché]

%TODO: Applichiamo il principio/proprietà di \emph{confusione} attraverso il passaggio 'Sub-bytes':
%TODO: itemize: 

%TODO: \textbf{Sub-bytes} oppure \underline{\emph{Sub-bytes}}: Ogni byte viene mappato in un diverso byte attraverso una sbox [o si scrive s-box?]. Questo passaggio/step applica la prima proprietà di \emph{confusione} di Shannon, perché oscura la relazione tra ogni byte.

%TODO: Applichiamo la proprietà di \emph{diffusione}:
%TODO: \textbf{Shift Rows} oppure \underline{\emph{Shift Rows}}: La seconda riga della matrice viene spostata di 1 verso sinistra. La terza riga di 2 posizioni e la quarta di 3.

%TODO: \textbf{Mix Columns} oppure \underline{\emph{Mix Columns}}: Ogni bits delle colonne della matrice (di stato) vengono mischiate.

%TODO: Dopodiché applichiamo la proprietà di \emph{segretezza della chiave}.
%TODO: \textbf{Add Round Key} oppure \underline{\emph{Add Round Key}}: Viene applicata la chiave del prossimo round attraverso uno XOR.

%TODO: Più rounds, più sicurezza, ma questo porterebbe ad un rallentamento delle performance.
%TODO: Per questo dobbiamo trovare/serve un compresso/un bilanciamento tra sicurezza e prestazioni.

%TODO: Quando AES era in sviluppo venne trovata una scorciatoia attraverso 6 rounds, [come possiamo notare ogni bit di output di un round dipende da ogni bit dei due rounds precedenti ] per evitare questo sono stati aggiunti 4 rounds extra, come \emph{margine di sicurezza}.

%TODO: [magari aggiungere qualche immagine sul procedimento]

%TODO: section/subsection: "Come può essere usato AES?" o "Con quali combinazioni può essere usato AES" o "In combinazione a cosa può essere usato AES"? o "Le modalità di AES"

%TODO: AES non può essere usato così com'è, ma necessita di essere utilizzato in combinazione a una modalità .
%TODO: Esistono varie modalità, qui ne elenco qualcuna:
%TODO: ECB (Electronic Code Book)
%TODO: CBC (Cipher Block Chaining)
%TODO: CFB (Cipher FeedBack)
%TODO: OFB (Output FeedBack)
%TODO: CTR (Counter mode)

%TODO: Approfondirli magari nella sezione dei pro e contro e dei possibili attacchi come: PA (Padding Attack), CPA (Chosen Plaintext Attack), CCA (Chosen Ci)

% -------------------------------- FINE CAPITOLO ---------------------------------------