% ================================ LA MATEMATICA DIETRO AES ===========================

\chapter{La Matematica dietro AES}

%TODO: matrici 4x4
%TODO: Campo finito (campo di Galois). (p^n elementi)
%TODO: trasformazione affine invertibile.
%TODO: punto fisso (in matematica), punti fissi opposti.

%TODO: S-Box
%TODO: Rijndael Key Scheduler/AES Key Scheduler

%TODO: numero colonne = numero bit blocco in input / 32

% ======================================================================================

%TODO: Parto spiegando che cos'è il campo finito di Galois GF(2^8)

%TODO: Cos'è un gruppo abelliano: gruppo con operazione con proprietà commutativa 
%TODO: Che cos'è un monoide: struttura algebrica con proprietà associativa e elemento neutro
%TODO (Che cos'è un gruppo)

%TODO: cos'è un anello: 3 assiomi degli anelli: gruppo abelliano rispetto all'addizione, monoide rispetto alla moltiplicazione, moltiplicazione distribuita rispetto all'addizione.
%TODO: spiegare omomorfismo e isomorfismo negli anelli?
%TODO: (Che cos'è un assioma)
%TODO: differenza tra campo e anello ed esempi a riguardo. (uno sull'anello e uno sui campi)
%TODO: anello finito: è un anello con un numero finito di elementi

%TODO: Teoria di Galois, Teorema di Abel-Ruffini?, Teorema fondamentale della Teoria di Galois?

%TODO: spiego le varie operazioni possibili in questo campo e come si eseguono:
%TODO: i numeri si possono rappresentare come polinomi, ovvero, per esempio: il numero esadecimale 0x53, ovvero 83 in decimale o 01010011 in binario può essere rappresentato come x^6 + x^4 + x + 1 perché 01010011 (1 * 2^6, 1 * 2^4, 1 * 2^1, 1 * 2^0 <=> x^6 + x^4 + x + 1)
%TODO: Somma e Sottrazione si fanno con lo XOR. 
%TODO: mostrare un esempio
%TODO: Moltiplicazione: modulo x^8 + x^4 + x^3 + x + 1 e il perché
%TODO: mostrare un esempio di moltiplicazione
%TODO: esponenziazione
%TODO: esempio o tabella degli esponenti
%TODO: Logaritmi
%TODO: esempio o tabella dei logaritmi
%TODO: Divisione
%TODO: esempio della divisione
%TODO: trovare il polinomio inverso

%TODO: spiego le varie operazioni dell'algoritmo come vengono eseguite in termini matematici.
%TODO: l's-box g e f => f(g(x))
%TODO: round constants nella key expansion
%TODO: Mix Columns: ogni colonna come polinomio, la moltiplico per un altro polinomio e recupero il resto dopo averla divisa per x^4 + 1, tutto questo è una moltiplicazione tra matrici.
%TODO: e altro.

% ---------------------------- SECTION: INTRODUZIONE ----------------------------------

%\newpage

\section{Introduzione}

% ---------------------------- SECTION:  ----------------------------------

\newpage

\section{} 

% -------------------------------- FINE CAPITOLO --------------------------------------