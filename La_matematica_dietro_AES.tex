% ================================ LA MATEMATICA DIETRO AES ===========================

\chapter{La Matematica dietro AES}

% ======================================================================================

%TODO: l's-box g e f => f(g(x))
%TODO: round constants nella key expansion
%TODO: Mix Columns: ogni colonna come polinomio, la moltiplico per un altro polinomio e recupero il resto dopo averla divisa per x^4 + 1, tutto questo è una moltiplicazione tra matrici.
%TODO: e altro.

%TODO: volendo aggiungere dei riferimenti, ovvero dei \label e \ref alla parti che spiego perché sono connesse.

%TODO: Aggiungere delle immagini volendo.

% ---------------------------- SECTION: INTRODUZIONE ----------------------------------

%\newpage

\section{Introduzione}

\index{Galois} \index{algoritmi} \index{campi} \index{anelli} \index{gruppi} \index{strutture} \index{campo}

\textsf{\small In questo capitolo tratteremo più a fondo la matematica che sta dietro questo algoritmo. Per poterlo fare avremo bisogno di introdurre alcune strutture matematiche come: i gruppi, gli anelli e i campi, dopodiché ci soffermeremo su uno specifico campo, quello di Galois, ne mostreremo le operazioni possibili e come ne usufruiremo nei passi trattati nel capitolo degli algoritmi.}

% ------------------------- SECTION: GRUPPI, ANELLI E CAMPI  --------------------------

\newpage

%TODO: volendo spiegare cos'è un assioma.

\section{Gruppi, Anelli e Campi}

\index{gruppi} \index{anelli} \index{campi}

\textsf{\small Prima di introdurre il campo di Galois, facciamo chiarezza tra i Gruppi, gli Anelli e i Campi.}

%TODO: o spiegare i gruppi, gli anelli e i campi prima del campo di Galois o dopo! Li metto in una sezione a parte o come subsections di "Il campo di Galois"?

\subsection{Gruppo}

\index{gruppo} \index{struttura} \index{algebrica} \index{struttura algebrica}

\textsf{\small Un gruppo è una struttura algebrica composta da un'insieme vuoto $\emptyset$ e un'operazione binaria ($+ \text{ e } \cdot$) per l'elemento neutro e per l'elemento inverso.}

\textsf{\small Il gruppo deve soddisfare tre assiomi:}

\begin{enumerate}[I.] %con package enumitem: [label=\Roman.]
	\item \textsf{\small proprietà associativa: dati a, b, c;  $(a \cdot b) \cdot c = a \cdot (b \cdot c)$}
	\item \textsf{\small esistenza dell'elemento neutro: elemento neutro \emph{e} tale che $a \cdot e = e \cdot a = a$}
	\item \textsf{\small esistenza dell'inverso: esiste a', detto inverso, tale che a $\cdot a' = a' \cdot a = e$}
\end{enumerate}

\subsubsection{Monoide}

\index{monoide} \index{struttura} \index{algebrica} \index{struttura algebrica}

\textsf{\small Il monoide è una struttura algebrica fornita dell'operazione binaria associativa e di un elemento neutro.}

\subsubsection{Gruppo abeliano}

\index{gruppo} \index{gruppo abeliano} \index{abeliano}

\textsf{\small  È un gruppo che gode della proprietà commutativa.}

\subsection{Anello}

\index{anello} \index{struttura} \index{algebrica} \index{struttura algebrica}

\textsf{\small Un anello è una struttura algebrica, dove sono definite due operazioni: $+ \text{ e } \cdot$ (somma e prodotto) che rispetta le proprietà: commutativa, distributiva (della moltiplicazione rispetto all'addizione) e dove esiste un elemento neutro nell'addizione ($a + 0 = 0 + a = a$) e nella moltiplicazione ($a \cdot 1 = 1 \cdot a = a$).}

\textsf{\small Quindi, un anello è: }

\index{monoide} \index{gruppo abeliano}

\begin{itemize}
	\item \textsf{\small  due operazioni binarie $+ \text{ e } \cdot$ che rispettano i 3 assiomi degli anelli:}
	\begin{itemize}
		\item \textsf{\small gruppo abeliano rispetto all'addizione (proprietà commutativa).}
		\item \textsf{\small monoide rispetto alla moltiplicazione (proprietà associativa ed elemento neutro)}
		\item \textsf{\small moltiplicazione distributiva rispetto all'addizione.}
	\end{itemize}
\end{itemize}

\subsubsection{Anello commutativo}

\index{anello}

\textsf{\small È un tipo di anello in cui l'operazione moltiplicativa gode della proprietà commutativa.}

\subsubsection{Anello finito}

\index{anello} \index{anello finito}

\textsf{\small Un anello finito è un anello che possiede un numero finito di elementi.}

\subsection{Campo}

\index{campo} \index{struttura} \index{algebrica} \index{struttura algebrica} 

\textsf{\small Un campo $\mathbb{K}$ è una struttura algebrica formata da un insieme non vuoto e due operazioni binarie $+ \text{ e } \cdot$ in cui:}

\index{insieme} \index{gruppo abeliano}

\begin{itemize}
	\item \textsf{\small l'insieme e l'operazione + sono un gruppo abeliano con elemento neutro.}
	\item \textsf{\small $\mathbb{K} \hspace{.1cm} \backslash \hspace{.1cm} \{0\}$ e l'operazione $\cdot$ sono un gruppo abeliano con elemento neutro 1.}
	\item \textsf{\small la moltiplicazione è distributiva rispetto all'addizione.}
\end{itemize}

\index{campo} \index{anello} \index{anello commutativo}

\textsf{\small Un campo può essere definito anche come un anello commutativo dove ogni elemento non nullo possiede un inverso.}
\textsf{\small O anche come un corpo commutativo alla moltiplicazione.}

\subsubsection{Campi vs Anelli} 

\index{campi} \index{anelli}

\textsf{\small La differenza fra queste due strutture è questa, ovvero:}

\index{gruppo}

\textsf{\small Un anello è un gruppo abeliano rispetto alla prima operazione (+ addizione), ma non rispetto alla seconda, mentre, il campo è un gruppo abeliano rispetto a entrambe le operazioni.}

\textsf{\small Di seguito, alcuni esempi:}

\index{anello} \index{gruppo abeliano}

\textsf{\small Esempio: L'anello dei numeri interi ($\mathbb{Z}, +, \cdot$) è un gruppo abeliano rispetto all'addizione: 5 + (-5) = 0. Ma non è un gruppo abeliano rispetto alla moltiplicazione, perché non esiste un inverso di un numero intero n tale che $n \cdot n^{-1} = 1, 5 \cdot \frac{1}{5} \underset{}{\notin \mathbb{Z}} = 1$}

\index{gruppo abeliano} \index{anello}

\textsf{\small Esempio: Anello numeri frazionari è un campo: $(\mathbb{Q}, +, \cdot)$ è un gruppo abeliano rispetto all'addizione e alla moltiplicazione: $5 + (-5) = 0. \hspace{.3cm} 5 \cdot \frac{1}{5} = 1$}

\subsection{Estensione di campi}

\index{estensione di campi} \index{campi} \index{coppie di campi}

\textsf{\small Essa è uno studio di coppie di campi, l'uno contenuto nell'altro. Una tale coppia prende il nome di \textbf{estensione di campi}. }

\index{sottocampi}

\textsf{\small Se L è un campo e K è un campo contenuto in L tale che le operazioni di campo in K sono le stesse di quelle in L, diciamo che K è un sottocampo di L, che L è un'estensione di K e che $L/K$ è un'\emph{estensione di campi}.}

\textsf{\small Fonte: Wikipedia \cite{wikipediaestensionedicampi}\relax }

\subsection{Estensione algebrica}

\index{estensione algebrica} \index{algebra astratta} \index{estensione di campi}

\textsf{\small L'estensione algebrica, in algebra astratta, una estensione di campi $L/K$ è detta algebrica se ogni elemento L è ottenibile come radice di un qualche polinomio a coefficienti in K.}

\textsf{\small Sia K un campo, una estensione è il dato di un altro campo L e di un omomorfismo iniettivo di K in L. Tramite l'omomorfismo K può essere visto come un sottocampo di L. L'estensione è generalmente indicata con la notazione $L/K$.}

\textsf{\small Fonte: Wikipedia \cite{wikipediaestensionealgebrica}\relax }

\subsection{Gruppo di Galois}

\index{gruppo di galois} \index{gruppo} \index{Galois}

\textsf{\small Il gruppo di Galois è un gruppo associato a un'estensione di campi, in particolare i gruppi associati a estensioni che sono di Galois.}

\textsf{\small Fonte: Wikipedia \cite{wikipediagruppodigalois}\relax }

\subsection{Estensione di Galois}

\index{estensione di galois} \index{Galois}

\textsf{\small È un'estensione algebrica $E/F$ che soddisfa le condizioni descritte qui sotto. Il senso è che un'estensione di Galois ha un gruppo di Galois e obbedisce al teorema fondamentale della teoria di Galois.}

\textsf{\small Fonte: Wikipedia \cite{wikipediaestensionedigalois}\relax }

% ---------------------------- SECTION: IL CAMPO DI GALOIS  ---------------------------

\section{Il campo di Galois}

\index{Galois}

\textsf{\small Il campo finito o anche denominato il campo di Galois, dal matematico francese Évariste Galois, è un campo che contiene un numero finito di elementi che ci permette di attuare operazioni su numeri a 8 bits.} %TODO: da riscrivere meglio

\index{teoria dei gruppi} \index{teoria di campi} \index{teoria di galois}

\textsf{\small Fornisce una connessione tra teoria di campi e teoria dei gruppi, attraverso il teorema fondamentale della teoria di galois.} %TODO: questo è da approfondire!

\textsf{\small Lo introdusse per studiare le radici dei polinomi e risolvere queste equazioni polinomiali in termini di gruppi di permutazione delle loro radici.}

\textsf{\small Un'equazione è risolvibile dai suoi radicali se le sue radici si possono esprimere da una formula che riguarda numeri interi, radici ennesime e le 4 operazioni aritmetiche basilari $(+, -, \cdot, /)$.}

\index{teorema di Abel-Ruffini} \index{Abel-Ruffini}

\textsf{\small Questo generalizza il teorema di Abel-Ruffini che afferma che non ci sono soluzioni in radicali alle equazioni polinomiali di grado 5 o superiori con coefficienti arbitrari.} %TODO: Questo generalizza il teorema di Abel-Ruffini che afferma che non c'è soluzioni in radicali alle equazioni polinomiali di grado 5 o superiori con coefficienti arbitrari. %TODO: questo è da mettere dopo l'introduzione al campo di Galois! E da semplificare.

\textsf{\small Ora arriviamo al motivo per cui ci interessa:} %TODO: Riscrivere meglio?

\index{Galois} \index{campo di Galois} \index{GF}

\textsf{\small Il più comune campo di Galois è il campo GF(256) o anche designato GF($2^8$) (GF = Galois Field) che ci permette di definire operazioni sono a livello di byte, dove i bits vengono interpretati come coefficienti dei polinomi del campo finito.}
%TODO: 

\subsection{Le operazioni del campo finito}

\index{campo finito} \index{campo}

\textsf{\small Nel campo finito sono disponibili le quattro operazioni basilari dell'aritmetica, $(+, -, \cdot, /)$ e di conseguenza l'esponenziazione e i logaritmi.} %TODO: riscrivere la parte "e di conseguenza"?

\textsf{\small I numeri vengono rappresentati come polinomi di 8 bits, ovvero, per esempio il numero esadecimale 0x53, ovvero 83 in decimale o 01010011 in binario può essere rappresentato come $x^6 + x^4 + x + 1$ perché 01010011 $(1 \cdot 2^6, 1 \cdot 2^4, 1 \cdot 2^1, 1 \cdot 2^0 <=> x^6 + x^4 + x + 1)$. }

%TODO: oppure scegliere un altro numero

\subsubsection{Addizione e sottrazione}

\index{addizione} \index{sottrazione} \index{somma}

%TODO: potrei utilizzare i gli $\underbrace{}$
\textsf{\small Somma e sottrazione vengono eseguite tramite uno XOR.}

\textsf{\small Per esempio: prendiamo due polinomi: $x + 1 \text{ e } x^2 + 1$.}
\textsf{\small $x + 1$ equivale a $2^1 + 2^0$, ovvero in binario il numero 11.}
\textsf{\small $x^2 + 1$ equivale a $1 \cdot 2^2 + 0 \cdot 2^1 + 1 \cdot 2^0$, ovvero in binario al numero 101.}
\textsf{\small dopodiché eseguiamo, semplicemente, uno XOR tra i due polinomi.}
\textsf{\small  $011 \oplus 101$ oppure $x + 1 \oplus x^2 + 1 = 110 \text{ o } x^2 + x$}

\subsubsection{Moltiplicazione}

\index{moltiplicazione}

\textsf{\small Per la moltiplicazione è un pochino più complicato.} %TODO: filino/è un pochino

\textsf{\small Avvenuta l'operazione, è necessario compiere un'operazione di modulo con il polinomio $x^8 + x^4 + x^3 + x + 1$. } %TODO: effettuare/compiere (magari spiegare il perché)

\textsf{\small per esempio $0x53 \cdot 0xCA = 0x01$, perché: } %TODO: cambiare numeri 0x53 e 0xCA

\textsf{\small $0x53 = 83 = 1010011 = x^6 + x^4 + x + 1$}

\textsf{\small $ 0xCA = 202 = 11001010 = x^7 + x^6 + x^3 + x $} \\

\textsf{\small $ (x^6 + x^4 + x + 1)(x^7 + x^6 + x^3 + x) $}

\textsf{\small $ = (x^{13} + x^{12} + x^9 + x^7)  + (x^{11} + x^{10} + x^7 + x^5) + (x^8 + x^7 + x^4 + x^2) + (x^7 + x^6 + x^3 + x)$}

\textsf{\small $ = x^{13} + x^{12} + x^{11} + x^{10} + x^9 + x^8 + x^6 + x^5 + x^4 + x^3 + x^2 + x$}

\textsf{\small $ x^{13} + x^{12} + x^{11} + x^{10} + x^9 + x^8 + x^6 + x^5 + x^4 + x^3 + x^2 + x \text{ modulo } x^8 + x^4 + x^3 + x + 1 = 1$.}

\subsubsection{Esponenziazione} %TODO: "Potenza"?

\index{esponenziazione} \index{tabella dei generatori}

\textsf{\small L'esponenziazione è semplicemente la ripetizione della moltiplicazione.}

\textsf{\small Alcuni numeri possiedono la proprietà di attraversare tutti i 255 numeri non-zero del campo quando vengono moltiplicati per se stessi. Questi numeri vengono chiamati \emph{generatori}.} %TODO: attraversare/essere in grado di attraversare; non-zero/non-zeri; denominati/chiamati

\textsf{\small Possiamo raccogliere questi numeri in una tabella, la tabella dei generatori.}

%TODO: volendo mostrare la tabella in un'immagine. MOSTRARE LA TABELLA.

\subsubsection{Logaritmi}

\index{logaritmi} \index{tabella dei generatori}

\textsf{\small Per eseguire i logaritmi possiamo avvalerci della tabella dei generatori.} %TODO: magari approfondire e fare anche un esempio.

%TODO: [...] creando così la tabella degli algoritmi.

\subsubsection{Divisione}

\index{divisione} \index{tabella dei generatori}

\textsf{\small Si fa prendendo i logaritmi dei due numeri e eseguedo il modulo 255.}

\textsf{\small $g^{(x - y) mod 255}$ dove g è uno dei generatori. }

%\subsubsection{Inverso} %TODO: Multiplication Inverse/Inverso Moltiplicativo

%TODO: uncomment/remove?

%TODO: 
%\textsf{\small }

% ----------- SECTION: IL TEOREMA FONDAMENTALE DELLA TEORIA DI GALOIS  ----------------

\section{Il teorema fondamentale della teoria di Galois} %TODO: oppure non in una section? oppure proprio non metterlo?

\index{teorema fondamentale della teoria di Galois} \index{Galois} \index{intercampi} \index{estensione di Galois}

\textsf{\small È un teorema che crea un legame tra gli intercampi di un'estensione di Galois e i sottogruppi del relativo gruppo di Galois.}

\textsf{\small Data un'estensione di campi $M/K$ con gruppo di Galois $G = \text{ Gal(M/K)}$ definiamo \emph{i} e \emph{j} (due funzioni): }

\begin{itemize}
	\item \textsf{\small Per ogni intercampo L (cioè tale che $K \subseteq L \subseteq M$) poniamo \emph{i}(L) = Gal(M/L), ossia il sottogruppo degli automorfismi di M che lasciano fissi gli elementi di L.}
	\item \textsf{\small Per ogni sottogruppo H di G, \emph{j}(H) che classicamente è indicato con $M^H$, è l'intercampo costituito dagli elementi di M lasciati fissi da tutti gli automorfismi di H.}
\end{itemize}

\textsf{\small Fonte: Wikipedia \cite{wikipediateoremafondamentaledigalois} } 

% -------------------------------- FINE CAPITOLO --------------------------------------