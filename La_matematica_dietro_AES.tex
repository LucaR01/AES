% ================================ LA MATEMATICA DIETRO AES ===========================

\chapter{La Matematica dietro AES}

%TODO: matrici 4x4
%TODO: Campo finito (campo di Galois). (p^n elementi)
%TODO: trasformazione affine invertibile.
%TODO: punto fisso (in matematica), punti fissi opposti.

%TODO: S-Box
%TODO: Rijndael Key Scheduler/AES Key Scheduler

%TODO: numero colonne = numero bit blocco in input / 32

% ======================================================================================

%TODO: 
%TODO: Parto spiegando che cos'è il campo finito di Galois GF(2^8)

%TODO: Cos'è un gruppo abelliano: gruppo con operazione con proprietà commutativa 
%TODO: Che cos'è un monoide: struttura algebrica con proprietà associativa e elemento neutro
%TODO (Che cos'è un gruppo)

%TODO: cos'è un anello: 3 assiomi degli anelli: gruppo abelliano rispetto all'addizione, monoide rispetto alla moltiplicazione, moltiplicazione distribuita rispetto all'addizione.
%TODO: spiegare omomorfismo e isomorfismo negli anelli?
%TODO: (Che cos'è un assioma)
%TODO: differenza tra campo e anello ed esempi a riguardo. (uno sull'anello e uno sui campi)
%TODO: anello finito: è un anello con un numero finito di elementi

%TODO: Teoria di Galois, Teorema di Abel-Ruffini?, Teorema fondamentale della Teoria di Galois?

%TODO: spiego le varie operazioni possibili in questo campo e come si eseguono:
%TODO: i numeri si possono rappresentare come polinomi, ovvero, per esempio: il numero esadecimale 0x53, ovvero 83 in decimale o 01010011 in binario può essere rappresentato come x^6 + x^4 + x + 1 perché 01010011 (1 * 2^6, 1 * 2^4, 1 * 2^1, 1 * 2^0 <=> x^6 + x^4 + x + 1)
%TODO: Somma e Sottrazione si fanno con lo XOR. 
%TODO: mostrare un esempio
%TODO: Moltiplicazione: modulo x^8 + x^4 + x^3 + x + 1 e il perché
%TODO: mostrare un esempio di moltiplicazione
%TODO: esponenziazione
%TODO: esempio o tabella degli esponenti
%TODO: Logaritmi
%TODO: esempio o tabella dei logaritmi
%TODO: Divisione
%TODO: esempio della divisione
%TODO: trovare il polinomio inverso

%TODO: spiego le varie operazioni dell'algoritmo come vengono eseguite in termini matematici.
%TODO: l's-box g e f => f(g(x))
%TODO: round constants nella key expansion
%TODO: Mix Columns: ogni colonna come polinomio, la moltiplico per un altro polinomio e recupero il resto dopo averla divisa per x^4 + 1, tutto questo è una moltiplicazione tra matrici.
%TODO: e altro.

% ---------------------------- SECTION: INTRODUZIONE ----------------------------------

%\newpage

\section{Introduzione}

%TODO: In questo capitolo tratteremo più a fondo la matematica che sta dietro questo algoritmo. Per poterlo fare avremo bisogno di introdurre alcune strutture matematiche come: i gruppi, gli anelli e i campi, dopodiché ci soffermeremo su uno/un specifico/determinato campo, quello di Galois, ne mostreremo le operazioni possibili e come vengono usufruite/utilizzate nei steps/passi mostrati/trattati nel capitolo degli algoritmi.
%TODO: da sistemare l'introduzione. 
\textsf{\small }

% ------------------------- SECTION: GRUPPI, ANELLI E CAMPI  --------------------------

\newpage

%TODO: volendo spiegare cos'è un assioma.

\section{Gruppi, Anelli e Campi} %TODO: oppure "Gruppi, Anelli e Campi: Le differenze" oppure "Campi, Gruppi e Anelli" o "Campi, Anelli e Gruppi".

\textsf{\small }

%TODO: Prima di introdurre il campo di Galois, facciamo chiarezza/differenziamo i/tra Gruppi, gli Anelli e i Campi.

%TODO: o spiegare i gruppi, gli anelli e i campi prima del campo di Galois o dopo! Li metto in una sezione a parte o come subsections di "Il campo di Galois"?

\subsection{Gruppo}

%TODO: Un gruppo è una struttura algebrica composta da un'insieme vuoto \emptyset e un'operazione binaria (+ e \cdot) per l'elemento neutro e per l'elemento inverso.
%TODO: Il gruppo deve soddisfare tre assiomi:
%TODO: itemize o enumerate
%TODO: I. proprietà associativa: dati a, b, c;  (a * b) * c = a * (b * c) %TODO: oppure usare x, y, z? Usare \cdot o *
%TODO: II. esistenza dell'elemento neutro: elemento neutro \emph{e} tale che a \cdot e = e \cdot a = a %TODO: a o alfa?
%TODO: III. esistenza dell'inverso: esiste a', detto inverso, tale che a \cdot a' = a' \cdot a = e
\textsf{\small }

\subsubsection{Monoide}

%TODO: Il monoide è una struttura algebrica fornita/dotata dell'operazione binaria associativa e di un elemento neutro.
\textsf{\small }

\subsubsection{Gruppo abelliano}

%TODO: È un gruppo che gode della proprietà commutativa.
\textsf{\small }

\subsection{Anello}

%TODO: Un anello è una struttura algebrica, dove sono definite due operazioni: + e \cdot (somma e prodotto) che rispetta le proprietà: commutativa, distributiva (della moltiplicazione rispetto all'addizione) e dove esiste un elemento neutro nell'addizione (a + 0 = 0 + a = a) e nella moltiplicazione (a \cdot 1 = 1 \cdot a = a).

%TODO: Quindi un anello è:
%TODO: itemize
%TODO: - 2 operazioni binarie + e \cdot che rispettano i 3 assiomi degli anelli:
%TODO: 	- gruppo abelliano rispetto all'addizione (proprietà commutativa).
%TODO: 	- monoide rispetto alla moltiplicazione (proprietà associativa ed elemento neutro)
%TODO: 	- moltiplicazione distributiva rispetto all'addizione.
\textsf{\small }

\subsubsection{Anello commutativo}

%TODO: È un tipo di anello in cui l'operazione moltiplicativa gode della proprietà commutativa.
\textsf{\small }

\subsubsection{Anello finito} %TODO: o altro oppure non serve come subsubsection

%TODO: Un anello finito è un anello che possiede un numero finito di elementi.
\textsf{\small }

\subsection{Campo} %TODO: o "Campi"?

%TODO: Un campo \mathbb{K} è una struttura algebrica formata da un insieme non vuoto e due operazioni binarie + e \cdot in cui:
%TODO: l'insieme e l'operazione + sono un gruppo abelliano con elemento neutro
%TODO: \mathbb{K} \ {0} e l'operazione \cdot sono un gruppo abelliano con elemento neutro 1.
%TODO: la moltiplicazione è distributiva rispetto all'addizione.

%TODO: Un campo può essere definito anche come un anello commutativo dove ogni elemento non nullo possiede un inverso.
%TODO: O anche come un corpo commutativo alla moltiplicazione. %TODO: volendo fare una subsection per il corpo?
\textsf{\small }

\subsubsection{Campi vs Anelli} %TODO: o "Campo vs Anello"? %TODO: forse questa parte non serve?

%TODO: La differenza fra queste due strutture è questa, ovvero:
%TODO: Un anello è un gruppo abelliano rispetto alla prima operazione (+ addizione), ma non rispetto alla seconda, mentre, il campo è un gruppo abelliano rispetto a entrambe le operazioni.
\textsf{\small }

%TODO: Di seguito, alcuni esempi:
%TOOD: Esempio: L'anello dei numeri interi (\mathbb{Z}, +, \cdot) è un gruppo abelliano rispetto all'addizione: 5 + (-5) = 0. Ma non è un gruppo abelliano rispetto alla moltiplicazione, perché non esiste un inverso di un numero intero n tale che n \cdot n^{-1} = 1, 5 * $\underset{\frac{1}{5}}{\notin \mathbb{Z}} = 1$

%TODO: Esempio: Anello numeri frazionari è un campo: $(\mathbb{Q}, +, \cdot)$ è un gruppo abelliano rispetto all'addizione e alla moltiplicazione: 5 + (-5) = 0. 5 \cdot $\frac{1}{5} = 1$

% ---------------------------- SECTION: IL CAMPO DI GALOIS  ---------------------------

\section{Il campo di Galois} %TODO: o "Il camppo finito"

%TODO: Il campo finito o anche denominato il campo di Galois, dal matematico francese Évariste Galois, è un campo che contiene un numero finito di elementi che ci permette di attuare operazioni su numeri a 8 bits. %TODO: da riscrivere meglio

%TODO: Fornisce una connessione tra teoria di campi e teoria dei gruppi, attraverso il teorema fondamentale della teoria di galois. %TODO: questo è da approfondire!

%TODO: Lo introdusse per studiare le radici dei polinomi e risolvere queste equazioni polinomiali in termini di gruppi di permutazione delle loro radici.
%TODO: Un'equazione è risolvibile dai suoi radicali se le sue radici si possono esprimere da una formula che riguarda numeri interi, radici ennesime e le 4 operazioni aritmetiche basilari (+, -, \cdot, /).
%TODO: Questo generalizza il teorema di Abel-Ruffini che afferma che non c'è soluzioni in radicali alle equazioni polinomiali di grado 5 o superiori con coefficienti arbitrari. %TODO: questo è da mettere dopo l'introduzione al campo di Galois! E da semplificare.

%TODO: Ora arriviamo al motivo per cui ci interessa:
%TODO: Il più comune campo di Galois è il campo GF(256) o anche scritto/definito GF(2^8) (GF = Galois Field) che ci permette di definire/dove le operazioni sono definite a livello di byte, dove i bits vengono interpretati come coefficienti dei polinomi del campo finito.

%TODO: Subsection? Le operazioni possibili nel campo di Galois
\subsection{Le operazioni del campo finito} %TODO: oppure "Le operazioni del campo di Galois" oppure "Le operazioni di GF(2^8)".

%TODO: Nel campo finito sono disponibili le quattro operazioni basilari dell'aritmetica, (+, -, \cdot, /) e i loro relativi./e di conseguenza l'esponenziazione e i logaritmi.

%TODO: I numeri vengono rappresentati come polinomi di 8 bits, ovvero, per esempio il numero esadecimale 0x53, ovvero 83 in decimale o 01010011 in binario può essere rappresentato come x^6 + x^4 + x + 1 perché 01010011 (1 * 2^6, 1 * 2^4, 1 * 2^1, 1 * 2^0 <=> x^6 + x^4 + x + 1). 

%TODO: oppure scegliere un altro numero!
\textsf{\small }

\subsubsection{Addizione e sottrazione} %TODO: oppure non serve fare una subsubsection?

%TODO: Somma e sottrazione vengono eseguite tramite uno XOR.
%TODO: Per esempio: prendiamo due polinomi: x + 1 e x^2 + 1.
%TODO: x + 1 equivale a 2^1 + 2^0, ovvero in binario il numero 11.
%TODO: x^2 + 1 equivale a 1 * 2^2 + 0 * 2^1 + 1 * 2^0, ovvero in binario al numero 101.
%TODO: dopodiché eseguiamo, semplicemente, uno XOR tra i due polinomi.
%TODO: 011 \xor 101 oppure x + 1 \xor x^2 + 1 = 110 o x^2 + x

%TODO: potrei utilizzare i gli $\underbrace{}$
\textsf{\small }

\subsubsection{Moltiplicazione}

%TODO: Per la moltiplicazione è un filino/è un pochino più complicato.
%TODO: Avvenuta l'operazione, è necessario effettuare/compiere un'operazioni di modulo con il polinomio x^8 + x^4 + x^3 + x + 1. (magari spiegare il perché)

%TODO: per esempio ... \cdot ... = ... , visto/perché: %TODO: ... vuol dire mettere il numero esadecimale.

\subsubsection{Esponenziazione} %TODO: "Potenza"?

\textsf{\small L'esponenziazione è semplicemente la ripetizione della moltiplicazione.}

%TODO: Alcuni numeri possiedono la proprietà di attraversare/essere in grado di attraversare tutti i 255 numeri non-zero/non-zeri del campo quando vengono moltiplicati per se stessi. Questi numeri vengono denominati/chiamati \emph{generatori}.

%TODO: Possiamo raccogliere questi numeri in una tabella, la tabella dei generatori.

%TODO: volendo mostrare la tabella in un'immagine.

\subsubsection{Logaritmi}

%\textsf{\small Per eseguire i logaritmi possiamo avvalerci della tabella dei generatori.} %TODO: magari approfondire e fare anche un esempio.

%TODO: [...] creando così la tabella degli algoritmi.

\subsubsection{Divisione}

%TODO: Si fa prendendo i logaritmi dei due numeri e eseguedo il modulo 255.
%TODO: g^{(x - y) mod 255} dove g è uno dei generatori. 
\textsf{\small }

\subsubsection{Inverso}

%TODO: 
\textsf{\small }

% ----------- SECTION: IL TEOREMA FONDAMENTALE DELLA TEORIA DI GALOIS  ----------------

\section{Il teorema fondamentale della teoria di Galois} %TODO: oppure non in una section? oppure proprio non metterlo.
\textsf{\small } 

% -------------------------------- FINE CAPITOLO --------------------------------------